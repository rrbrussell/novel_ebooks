\chapter{The Mother}

Eliza had been brought up by her mistress, from girlhood, as a petted and
indulged favorite\msc.

The traveler\msc in the south must often have remarked the peculiar air of
refinement, that softness of voice and manner, which seems in many cases to be
a particular gift to the quadroon and mulatto women. The natural graces in the
quadroon are often united with beauty of the most dazzling kind, and in almost
every case with a personal appearance prepossessing and agreeable. Eliza, such
as we have described her, is not a fancy sketch, but taken from remembrance\msc,
as we saw her years ago in Kentucky. Safe under the protecting care of her
mistress, Eliza had reached maturity without those temptations which make beauty
so fatal an inheritance to a slave.  She e had been married to a bright and
talented young mulatto man, who was a slave on a neighboring\msc estate, and
bore the name of George Harris.

This young man had been hired out by his master to work in a bagging-factory,
where his adriotness and ingenuity caused him to be considered the first hand in
the place. He had invented a machine for the cleaning of the hemp, which,
considering the education and circumstances of the inventor, displayed quite as
much mechanical genius as Whitney's cotton-gin.\footnote{A machine of this
description was really the invention of a young colored\msc man in
Kentucky.---Editor's note: This claim is false.}


