Chapter XVI.
THE RETURN HOME


By three bells that morning they were all stirring their stumps; for
there was a big sea running; and Tootles, the bo'sun, was among them,
with a rope's end in his hand and chewing tobacco. They all donned
pirate clothes cut off at the knee, shaved smartly, and tumbled up,
with the true nautical roll and hitching their trousers.

It need not be said who was the captain. Nibs and John were first and
second mate. There was a woman aboard. The rest were tars before the
mast, and lived in the fo'c'sle. Peter had already lashed himself to
the wheel; but he piped all hands and delivered a short address to
them; said he hoped they would do their duty like gallant hearties, but
that he knew they were the scum of Rio and the Gold Coast, and if they
snapped at him he would tear them. The bluff strident words struck the
note sailors understood, and they cheered him lustily. Then a few sharp
orders were given, and they turned the ship round, and nosed her for
the mainland.

Captain Pan calculated, after consulting the ship's chart, that if this
weather lasted they should strike the Azores about the 21st of June,
after which it would save time to fly.

Some of them wanted it to be an honest ship and others were in favour
of keeping it a pirate; but the captain treated them as dogs, and they
dared not express their wishes to him even in a round robin. Instant
obedience was the only safe thing. Slightly got a dozen for looking
perplexed when told to take soundings. The general feeling was that
Peter was honest just now to lull Wendy's suspicions, but that there
might be a change when the new suit was ready, which, against her will,
she was making for him out of some of Hook's wickedest garments. It was
afterwards whispered among them that on the first night he wore this
suit he sat long in the cabin with Hook's cigar-holder in his mouth and
one hand clenched, all but for the forefinger, which he bent and held
threateningly aloft like a hook.

Instead of watching the ship, however, we must now return to that
desolate home from which three of our characters had taken heartless
flight so long ago. It seems a shame to have neglected No. 14 all this
time; and yet we may be sure that Mrs. Darling does not blame us. If we
had returned sooner to look with sorrowful sympathy at her, she would
probably have cried, ``Don't be silly; what do I matter? Do go back and
keep an eye on the children.'' So long as mothers are like this their
children will take advantage of them; and they may lay to that.

Even now we venture into that familiar nursery only because its lawful
occupants are on their way home; we are merely hurrying on in advance
of them to see that their beds are properly aired and that Mr. and Mrs.
Darling do not go out for the evening. We are no more than servants.
Why on earth should their beds be properly aired, seeing that they left
them in such a thankless hurry? Would it not serve them jolly well
right if they came back and found that their parents were spending the
week-end in the country? It would be the moral lesson they have been in
need of ever since we met them; but if we contrived things in this way
Mrs. Darling would never forgive us.

One thing I should like to do immensely, and that is to tell her, in
the way authors have, that the children are coming back, that indeed
they will be here on Thursday week. This would spoil so completely the
surprise to which Wendy and John and Michael are looking forward. They
have been planning it out on the ship: mother's rapture, father's shout
of joy, Nana's leap through the air to embrace them first, when what
they ought to be prepared for is a good hiding. How delicious to spoil
it all by breaking the news in advance; so that when they enter grandly
Mrs. Darling may not even offer Wendy her mouth, and Mr. Darling may
exclaim pettishly, ``Dash it all, here are those boys again.'' However,
we should get no thanks even for this. We are beginning to know Mrs.
Darling by this time, and may be sure that she would upbraid us for
depriving the children of their little pleasure.

``But, my dear madam, it is ten days till Thursday week; so that by
telling you what's what, we can save you ten days of unhappiness.''

``Yes, but at what a cost! By depriving the children of ten minutes of
delight.''

``Oh, if you look at it in that way!''

``What other way is there in which to look at it?''

You see, the woman had no proper spirit. I had meant to say
extraordinarily nice things about her; but I despise her, and not one
of them will I say now. She does not really need to be told to have
things ready, for they are ready. All the beds are aired, and she never
leaves the house, and observe, the window is open. For all the use we
are to her, we might well go back to the ship. However, as we are here
we may as well stay and look on. That is all we are, lookers-on. Nobody
really wants us. So let us watch and say jaggy things, in the hope that
some of them will hurt.

The only change to be seen in the night-nursery is that between nine
and six the kennel is no longer there. When the children flew away, Mr.
Darling felt in his bones that all the blame was his for having chained
Nana up, and that from first to last she had been wiser than he. Of
course, as we have seen, he was quite a simple man; indeed he might
have passed for a boy again if he had been able to take his baldness
off; but he had also a noble sense of justice and a lion's courage to
do what seemed right to him; and having thought the matter out with
anxious care after the flight of the children, he went down on all
fours and crawled into the kennel. To all Mrs. Darling's dear
invitations to him to come out he replied sadly but firmly:

``No, my own one, this is the place for me.''

In the bitterness of his remorse he swore that he would never leave the
kennel until his children came back. Of course this was a pity; but
whatever Mr. Darling did he had to do in excess, otherwise he soon gave
up doing it. And there never was a more humble man than the once proud
George Darling, as he sat in the kennel of an evening talking with his
wife of their children and all their pretty ways.

Very touching was his deference to Nana. He would not let her come into
the kennel, but on all other matters he followed her wishes implicitly.

Every morning the kennel was carried with Mr. Darling in it to a cab,
which conveyed him to his office, and he returned home in the same way
at six. Something of the strength of character of the man will be seen
if we remember how sensitive he was to the opinion of neighbours: this
man whose every movement now attracted surprised attention. Inwardly he
must have suffered torture; but he preserved a calm exterior even when
the young criticised his little home, and he always lifted his hat
courteously to any lady who looked inside.

It may have been Quixotic, but it was magnificent. Soon the inward
meaning of it leaked out, and the great heart of the public was
touched. Crowds followed the cab, cheering it lustily; charming girls
scaled it to get his autograph; interviews appeared in the better class
of papers, and society invited him to dinner and added, ``Do come in the
kennel.''

On that eventful Thursday week, Mrs. Darling was in the night-nursery
awaiting George's return home; a very sad-eyed woman. Now that we look
at her closely and remember the gaiety of her in the old days, all gone
now just because she has lost her babes, I find I won't be able to say
nasty things about her after all. If she was too fond of her rubbishy
children, she couldn't help it. Look at her in her chair, where she has
fallen asleep. The corner of her mouth, where one looks first, is
almost withered up. Her hand moves restlessly on her breast as if she
had a pain there. Some like Peter best, and some like Wendy best, but I
like her best. Suppose, to make her happy, we whisper to her in her
sleep that the brats are coming back. They are really within two miles
of the window now, and flying strong, but all we need whisper is that
they are on the way. Let's.

It is a pity we did it, for she has started up, calling their names;
and there is no one in the room but Nana.

``O Nana, I dreamt my dear ones had come back.''

Nana had filmy eyes, but all she could do was put her paw gently on her
mistress's lap; and they were sitting together thus when the kennel was
brought back. As Mr. Darling puts his head out to kiss his wife, we see
that his face is more worn than of yore, but has a softer expression.

He gave his hat to Liza, who took it scornfully; for she had no
imagination, and was quite incapable of understanding the motives of
such a man. Outside, the crowd who had accompanied the cab home were
still cheering, and he was naturally not unmoved.

``Listen to them,'' he said; ``it is very gratifying.''

``Lots of little boys,'' sneered Liza.

``There were several adults to-day,'' he assured her with a faint flush;
but when she tossed her head he had not a word of reproof for her.
Social success had not spoilt him; it had made him sweeter. For some
time he sat with his head out of the kennel, talking with Mrs. Darling
of this success, and pressing her hand reassuringly when she said she
hoped his head would not be turned by it.

``But if I had been a weak man,'' he said. ``Good heavens, if I had been a
weak man!''

``And, George,'' she said timidly, ``you are as full of remorse as ever,
aren't you?''

``Full of remorse as ever, dearest! See my punishment: living in a
kennel.''

``But it is punishment, isn't it, George? You are sure you are not
enjoying it?''

``My love!''

You may be sure she begged his pardon; and then, feeling drowsy, he
curled round in the kennel.

``Won't you play me to sleep,'' he asked, ``on the nursery piano?'' and as
she was crossing to the day-nursery he added thoughtlessly, ``And shut
that window. I feel a draught.''

``O George, never ask me to do that. The window must always be left open
for them, always, always.''

Now it was his turn to beg her pardon; and she went into the
day-nursery and played, and soon he was asleep; and while he slept,
Wendy and John and Michael flew into the room.

Oh no. We have written it so, because that was the charming arrangement
planned by them before we left the ship; but something must have
happened since then, for it is not they who have flown in, it is Peter
and Tinker Bell.

Peter's first words tell all.

``Quick Tink,'' he whispered, ``close the window; bar it! That's right.
Now you and I must get away by the door; and when Wendy comes she will
think her mother has barred her out; and she will have to go back with
me.''

Now I understand what had hitherto puzzled me, why when Peter had
exterminated the pirates he did not return to the island and leave Tink
to escort the children to the mainland. This trick had been in his head
all the time.

Instead of feeling that he was behaving badly he danced with glee; then
he peeped into the day-nursery to see who was playing. He whispered to
Tink, ``It's Wendy's mother! She is a pretty lady, but not so pretty as
my mother. Her mouth is full of thimbles, but not so full as my
mother's was.''

Of course he knew nothing whatever about his mother; but he sometimes
bragged about her.

He did not know the tune, which was ``Home, Sweet Home,'' but he knew it
was saying, ``Come back, Wendy, Wendy, Wendy''; and he cried exultantly,
``You will never see Wendy again, lady, for the window is barred!''

He peeped in again to see why the music had stopped, and now he saw
that Mrs. Darling had laid her head on the box, and that two tears were
sitting on her eyes.

``She wants me to unbar the window,'' thought Peter, ``but I won't, not
I!''

He peeped again, and the tears were still there, or another two had
taken their place.

``She's awfully fond of Wendy,'' he said to himself. He was angry with
her now for not seeing why she could not have Wendy.

The reason was so simple: ``I'm fond of her too. We can't both have her,
lady.''

But the lady would not make the best of it, and he was unhappy. He
ceased to look at her, but even then she would not let go of him. He
skipped about and made funny faces, but when he stopped it was just as
if she were inside him, knocking.

``Oh, all right,'' he said at last, and gulped. Then he unbarred the
window. ``Come on, Tink,'' he cried, with a frightful sneer at the laws
of nature; ``we don't want any silly mothers;'' and he flew away.

Thus Wendy and John and Michael found the window open for them after
all, which of course was more than they deserved. They alighted on the
floor, quite unashamed of themselves, and the youngest one had already
forgotten his home.

``John,'' he said, looking around him doubtfully, ``I think I have been
here before.''

``Of course you have, you silly. There is your old bed.''

``So it is,'' Michael said, but not with much conviction.

``I say,'' cried John, ``the kennel!'' and he dashed across to look into
it.

``Perhaps Nana is inside it,'' Wendy said.

But John whistled. ``Hullo,'' he said, ``there's a man inside it.''

``It's father!'' exclaimed Wendy.

``Let me see father,'' Michael begged eagerly, and he took a good look.
``He is not so big as the pirate I killed,'' he said with such frank
disappointment that I am glad Mr. Darling was asleep; it would have
been sad if those had been the first words he heard his little Michael
say.

Wendy and John had been taken aback somewhat at finding their father in
the kennel.

``Surely,'' said John, like one who had lost faith in his memory, ``he
used not to sleep in the kennel?''

``John,'' Wendy said falteringly, ``perhaps we don't remember the old life
as well as we thought we did.''

A chill fell upon them; and serve them right.

``It is very careless of mother,'' said that young scoundrel John, ``not
to be here when we come back.''

It was then that Mrs. Darling began playing again.

``It's mother!'' cried Wendy, peeping.

``So it is!'' said John.

``Then are you not really our mother, Wendy?'' asked Michael, who was
surely sleepy.

``Oh dear!'' exclaimed Wendy, with her first real twinge of remorse, ``it
was quite time we came back.''

``Let us creep in,'' John suggested, ``and put our hands over her eyes.''

But Wendy, who saw that they must break the joyous news more gently,
had a better plan.

``Let us all slip into our beds, and be there when she comes in, just as
if we had never been away.''

And so when Mrs. Darling went back to the night-nursery to see if her
husband was asleep, all the beds were occupied. The children waited for
her cry of joy, but it did not come. She saw them, but she did not
believe they were there. You see, she saw them in their beds so often
in her dreams that she thought this was just the dream hanging around
her still.

She sat down in the chair by the fire, where in the old days she had
nursed them.

They could not understand this, and a cold fear fell upon all the three
of them.

``Mother!'' Wendy cried.

``That's Wendy,'' she said, but still she was sure it was the dream.

``Mother!''

``That's John,'' she said.

``Mother!'' cried Michael. He knew her now.

``That's Michael,'' she said, and she stretched out her arms for the
three little selfish children they would never envelop again. Yes, they
did, they went round Wendy and John and Michael, who had slipped out of
bed and run to her.

``George, George!'' she cried when she could speak; and Mr. Darling woke
to share her bliss, and Nana came rushing in. There could not have been
a lovelier sight; but there was none to see it except a little boy who
was staring in at the window. He had had ecstasies innumerable that
other children can never know; but he was looking through the window at
the one joy from which he must be for ever barred.