Chapter V.
THE ISLAND COME TRUE


Feeling that Peter was on his way back, the Neverland had again woke
into life. We ought to use the pluperfect and say wakened, but woke is
better and was always used by Peter.

In his absence things are usually quiet on the island. The fairies take
an hour longer in the morning, the beasts attend to their young, the
redskins feed heavily for six days and nights, and when pirates and
lost boys meet they merely bite their thumbs at each other. But with
the coming of Peter, who hates lethargy, they are under way again: if
you put your ear to the ground now, you would hear the whole island
seething with life.

On this evening the chief forces of the island were disposed as
follows. The lost boys were out looking for Peter, the pirates were out
looking for the lost boys, the redskins were out looking for the
pirates, and the beasts were out looking for the redskins. They were
going round and round the island, but they did not meet because all
were going at the same rate.

All wanted blood except the boys, who liked it as a rule, but to-night
were out to greet their captain. The boys on the island vary, of
course, in numbers, according as they get killed and so on; and when
they seem to be growing up, which is against the rules, Peter thins
them out; but at this time there were six of them, counting the twins
as two. Let us pretend to lie here among the sugar-cane and watch them
as they steal by in single file, each with his hand on his dagger.

They are forbidden by Peter to look in the least like him, and they
wear the skins of the bears slain by themselves, in which they are so
round and furry that when they fall they roll. They have therefore
become very sure-footed.

The first to pass is Tootles, not the least brave but the most
unfortunate of all that gallant band. He had been in fewer adventures
than any of them, because the big things constantly happened just when
he had stepped round the corner; all would be quiet, he would take the
opportunity of going off to gather a few sticks for firewood, and then
when he returned the others would be sweeping up the blood. This
ill-luck had given a gentle melancholy to his countenance, but instead
of souring his nature had sweetened it, so that he was quite the
humblest of the boys. Poor kind Tootles, there is danger in the air for
you to-night. Take care lest an adventure is now offered you, which, if
accepted, will plunge you in deepest woe. Tootles, the fairy Tink, who
is bent on mischief this night is looking for a tool, and she thinks
you are the most easily tricked of the boys. 'Ware Tinker Bell.

Would that he could hear us, but we are not really on the island, and
he passes by, biting his knuckles.

Next comes Nibs, the gay and debonair, followed by Slightly, who cuts
whistles out of the trees and dances ecstatically to his own tunes.
Slightly is the most conceited of the boys. He thinks he remembers the
days before he was lost, with their manners and customs, and this has
given his nose an offensive tilt. Curly is fourth; he is a pickle, and
so often has he had to deliver up his person when Peter said sternly,
``Stand forth the one who did this thing,'' that now at the command he
stands forth automatically whether he has done it or not. Last come the
Twins, who cannot be described because we should be sure to be
describing the wrong one. Peter never quite knew what twins were, and
his band were not allowed to know anything he did not know, so these
two were always vague about themselves, and did their best to give
satisfaction by keeping close together in an apologetic sort of way.

The boys vanish in the gloom, and after a pause, but not a long pause,
for things go briskly on the island, come the pirates on their track.
We hear them before they are seen, and it is always the same dreadful
song:

``Avast belay, yo ho, heave to,
    A-pirating we go,
And if we're parted by a shot
    We're sure to meet below!''

A more villainous-looking lot never hung in a row on Execution dock.
Here, a little in advance, ever and again with his head to the ground
listening, his great arms bare, pieces of eight in his ears as
ornaments, is the handsome Italian Cecco, who cut his name in letters
of blood on the back of the governor of the prison at Gao. That
gigantic black behind him has had many names since he dropped the one
with which dusky mothers still terrify their children on the banks of
the Guadjo-mo. Here is Bill Jukes, every inch of him tattooed, the same
Bill Jukes who got six dozen on the _Walrus_ from Flint before he would
drop the bag of moidores; and Cookson, said to be Black Murphy's
brother (but this was never proved), and Gentleman Starkey, once an
usher in a public school and still dainty in his ways of killing; and
Skylights (Morgan's Skylights); and the Irish bo'sun Smee, an oddly
genial man who stabbed, so to speak, without offence, and was the only
Non-conformist in Hook's crew; and Noodler, whose hands were fixed on
backwards; and Robt. Mullins and Alf Mason and many another ruffian
long known and feared on the Spanish Main.

In the midst of them, the blackest and largest in that dark setting,
reclined James Hook, or as he wrote himself, Jas. Hook, of whom it is
said he was the only man that the Sea-Cook feared. He lay at his ease
in a rough chariot drawn and propelled by his men, and instead of a
right hand he had the iron hook with which ever and anon he encouraged
them to increase their pace. As dogs this terrible man treated and
addressed them, and as dogs they obeyed him. In person he was
cadaverous and blackavized, and his hair was dressed in long curls,
which at a little distance looked like black candles, and gave a
singularly threatening expression to his handsome countenance. His eyes
were of the blue of the forget-me-not, and of a profound melancholy,
save when he was plunging his hook into you, at which time two red
spots appeared in them and lit them up horribly. In manner, something
of the grand seigneur still clung to him, so that he even ripped you up
with an air, and I have been told that he was a _raconteur_ of repute.
He was never more sinister than when he was most polite, which is
probably the truest test of breeding; and the elegance of his diction,
even when he was swearing, no less than the distinction of his
demeanour, showed him one of a different cast from his crew. A man of
indomitable courage, it was said that the only thing he shied at was
the sight of his own blood, which was thick and of an unusual colour.
In dress he somewhat aped the attire associated with the name of
Charles II, having heard it said in some earlier period of his career
that he bore a strange resemblance to the ill-fated Stuarts; and in his
mouth he had a holder of his own contrivance which enabled him to smoke
two cigars at once. But undoubtedly the grimmest part of him was his
iron claw.

Let us now kill a pirate, to show Hook's method. Skylights will do. As
they pass, Skylights lurches clumsily against him, ruffling his lace
collar; the hook shoots forth, there is a tearing sound and one
screech, then the body is kicked aside, and the pirates pass on. He has
not even taken the cigars from his mouth.

Such is the terrible man against whom Peter Pan is pitted. Which will
win?

On the trail of the pirates, stealing noiselessly down the war-path,
which is not visible to inexperienced eyes, come the redskins, every
one of them with his eyes peeled. They carry tomahawks and knives, and
their naked bodies gleam with paint and oil. Strung around them are
scalps, of boys as well as of pirates, for these are the Piccaninny
tribe, and not to be confused with the softer-hearted Delawares or the
Hurons. In the van, on all fours, is Great Big Little Panther, a brave
of so many scalps that in his present position they somewhat impede his
progress. Bringing up the rear, the place of greatest danger, comes
Tiger Lily, proudly erect, a princess in her own right. She is the most
beautiful of dusky Dianas and the belle of the Piccaninnies,
coquettish, cold and amorous by turns; there is not a brave who would
not have the wayward thing to wife, but she staves off the altar with a
hatchet. Observe how they pass over fallen twigs without making the
slightest noise. The only sound to be heard is their somewhat heavy
breathing. The fact is that they are all a little fat just now after
the heavy gorging, but in time they will work this off. For the moment,
however, it constitutes their chief danger.

The redskins disappear as they have come like shadows, and soon their
place is taken by the beasts, a great and motley procession: lions,
tigers, bears, and the innumerable smaller savage things that flee from
them, for every kind of beast, and, more particularly, all the
man-eaters, live cheek by jowl on the favoured island. Their tongues
are hanging out, they are hungry to-night.

When they have passed, comes the last figure of all, a gigantic
crocodile. We shall see for whom she is looking presently.

The crocodile passes, but soon the boys appear again, for the
procession must continue indefinitely until one of the parties stops or
changes its pace. Then quickly they will be on top of each other.

All are keeping a sharp look-out in front, but none suspects that the
danger may be creeping up from behind. This shows how real the island
was.

The first to fall out of the moving circle was the boys. They flung
themselves down on the sward, close to their underground home.

``I do wish Peter would come back,'' every one of them said nervously,
though in height and still more in breadth they were all larger than
their captain.

``I am the only one who is not afraid of the pirates,'' Slightly said, in
the tone that prevented his being a general favourite; but perhaps some
distant sound disturbed him, for he added hastily, ``but I wish he would
come back, and tell us whether he has heard anything more about
Cinderella.''

They talked of Cinderella, and Tootles was confident that his mother
must have been very like her.

It was only in Peter's absence that they could speak of mothers, the
subject being forbidden by him as silly.

``All I remember about my mother,'' Nibs told them, ``is that she often
said to my father, ‘Oh, how I wish I had a cheque-book of my own!' I
don't know what a cheque-book is, but I should just love to give my
mother one.''

While they talked they heard a distant sound. You or I, not being wild
things of the woods, would have heard nothing, but they heard it, and
it was the grim song:

``Yo ho, yo ho, the pirate life,
    The flag o' skull and bones,
A merry hour, a hempen rope,
    And hey for Davy Jones.''

At once the lost boys—but where are they? They are no longer there.
Rabbits could not have disappeared more quickly.

I will tell you where they are. With the exception of Nibs, who has
darted away to reconnoitre, they are already in their home under the
ground, a very delightful residence of which we shall see a good deal
presently. But how have they reached it? for there is no entrance to be
seen, not so much as a large stone, which if rolled away, would
disclose the mouth of a cave. Look closely, however, and you may note
that there are here seven large trees, each with a hole in its hollow
trunk as large as a boy. These are the seven entrances to the home
under the ground, for which Hook has been searching in vain these many
moons. Will he find it tonight?

As the pirates advanced, the quick eye of Starkey sighted Nibs
disappearing through the wood, and at once his pistol flashed out. But
an iron claw gripped his shoulder.

``Captain, let go!'' he cried, writhing.

Now for the first time we hear the voice of Hook. It was a black voice.
``Put back that pistol first,'' it said threateningly.

``It was one of those boys you hate. I could have shot him dead.''

``Ay, and the sound would have brought Tiger Lily's redskins upon us. Do
you want to lose your scalp?''

``Shall I after him, Captain,'' asked pathetic Smee, ``and tickle him with
Johnny Corkscrew?'' Smee had pleasant names for everything, and his
cutlass was Johnny Corkscrew, because he wiggled it in the wound. One
could mention many lovable traits in Smee. For instance, after killing,
it was his spectacles he wiped instead of his weapon.

``Johnny's a silent fellow,'' he reminded Hook.

``Not now, Smee,'' Hook said darkly. ``He is only one, and I want to
mischief all the seven. Scatter and look for them.''

The pirates disappeared among the trees, and in a moment their Captain
and Smee were alone. Hook heaved a heavy sigh, and I know not why it
was, perhaps it was because of the soft beauty of the evening, but
there came over him a desire to confide to his faithful bo'sun the
story of his life. He spoke long and earnestly, but what it was all
about Smee, who was rather stupid, did not know in the least.

Anon he caught the word Peter.

``Most of all,'' Hook was saying passionately, ``I want their captain,
Peter Pan. 'Twas he cut off my arm.'' He brandished the hook
threateningly. ``I've waited long to shake his hand with this. Oh, I'll
tear him!''

``And yet,'' said Smee, ``I have often heard you say that hook was worth a
score of hands, for combing the hair and other homely uses.''

``Ay,'' the captain answered, ``if I was a mother I would pray to have my
children born with this instead of that,'' and he cast a look of pride
upon his iron hand and one of scorn upon the other. Then again he
frowned.

``Peter flung my arm,'' he said, wincing, ``to a crocodile that happened
to be passing by.''

``I have often,'' said Smee, ``noticed your strange dread of crocodiles.''

``Not of crocodiles,'' Hook corrected him, ``but of that one crocodile.''
He lowered his voice. ``It liked my arm so much, Smee, that it has
followed me ever since, from sea to sea and from land to land, licking
its lips for the rest of me.''

``In a way,'' said Smee, ``it's sort of a compliment.''

``I want no such compliments,'' Hook barked petulantly. ``I want Peter
Pan, who first gave the brute its taste for me.''

He sat down on a large mushroom, and now there was a quiver in his
voice. ``Smee,'' he said huskily, ``that crocodile would have had me
before this, but by a lucky chance it swallowed a clock which goes tick
tick inside it, and so before it can reach me I hear the tick and
bolt.'' He laughed, but in a hollow way.

``Some day,'' said Smee, ``the clock will run down, and then he'll get
you.''

Hook wetted his dry lips. ``Ay,'' he said, ``that's the fear that haunts
me.''

Since sitting down he had felt curiously warm. ``Smee,'' he said, ``this
seat is hot.'' He jumped up. ``Odds bobs, hammer and tongs I'm burning.''

They examined the mushroom, which was of a size and solidity unknown on
the mainland; they tried to pull it up, and it came away at once in
their hands, for it had no root. Stranger still, smoke began at once to
ascend. The pirates looked at each other. ``A chimney!'' they both
exclaimed.

They had indeed discovered the chimney of the home under the ground. It
was the custom of the boys to stop it with a mushroom when enemies were
in the neighbourhood.

Not only smoke came out of it. There came also children's voices, for
so safe did the boys feel in their hiding-place that they were gaily
chattering. The pirates listened grimly, and then replaced the
mushroom. They looked around them and noted the holes in the seven
trees.

``Did you hear them say Peter Pan's from home?'' Smee whispered,
fidgeting with Johnny Corkscrew.

Hook nodded. He stood for a long time lost in thought, and at last a
curdling smile lit up his swarthy face. Smee had been waiting for it.
``Unrip your plan, captain,'' he cried eagerly.

``To return to the ship,'' Hook replied slowly through his teeth, ``and
cook a large rich cake of a jolly thickness with green sugar on it.
There can be but one room below, for there is but one chimney. The
silly moles had not the sense to see that they did not need a door
apiece. That shows they have no mother. We will leave the cake on the
shore of the Mermaids' Lagoon. These boys are always swimming about
there, playing with the mermaids. They will find the cake and they will
gobble it up, because, having no mother, they don't know how dangerous
'tis to eat rich damp cake.'' He burst into laughter, not hollow
laughter now, but honest laughter. ``Aha, they will die.''

Smee had listened with growing admiration.

``It's the wickedest, prettiest policy ever I heard of!'' he cried, and
in their exultation they danced and sang:

``Avast, belay, when I appear,
    By fear they're overtook;
Nought's left upon your bones when you
    Have shaken claws with Hook.''

They began the verse, but they never finished it, for another sound
broke in and stilled them. There was at first such a tiny sound that a
leaf might have fallen on it and smothered it, but as it came nearer it
was more distinct.

Tick tick tick tick!

Hook stood shuddering, one foot in the air.

``The crocodile!'' he gasped, and bounded away, followed by his bo'sun.

It was indeed the crocodile. It had passed the redskins, who were now
on the trail of the other pirates. It oozed on after Hook.

Once more the boys emerged into the open; but the dangers of the night
were not yet over, for presently Nibs rushed breathless into their
midst, pursued by a pack of wolves. The tongues of the pursuers were
hanging out; the baying of them was horrible.

``Save me, save me!'' cried Nibs, falling on the ground.

``But what can we do, what can we do?''

It was a high compliment to Peter that at that dire moment their
thoughts turned to him.

``What would Peter do?'' they cried simultaneously.

Almost in the same breath they cried, ``Peter would look at them through
his legs.''

And then, ``Let us do what Peter would do.''

It is quite the most successful way of defying wolves, and as one boy
they bent and looked through their legs. The next moment is the long
one, but victory came quickly, for as the boys advanced upon them in
the terrible attitude, the wolves dropped their tails and fled.

Now Nibs rose from the ground, and the others thought that his staring
eyes still saw the wolves. But it was not wolves he saw.

``I have seen a wonderfuller thing,'' he cried, as they gathered round
him eagerly. ``A great white bird. It is flying this way.''

``What kind of a bird, do you think?''

``I don't know,'' Nibs said, awestruck, ``but it looks so weary, and as it
flies it moans, ‘Poor Wendy.'''

``Poor Wendy?''

``I remember,'' said Slightly instantly, ``there are birds called
Wendies.''

``See, it comes!'' cried Curly, pointing to Wendy in the heavens.

Wendy was now almost overhead, and they could hear her plaintive cry.
But more distinct came the shrill voice of Tinker Bell. The jealous
fairy had now cast off all disguise of friendship, and was darting at
her victim from every direction, pinching savagely each time she
touched.

``Hullo, Tink,'' cried the wondering boys.

Tink's reply rang out: ``Peter wants you to shoot the Wendy.''

It was not in their nature to question when Peter ordered. ``Let us do
what Peter wishes!'' cried the simple boys. ``Quick, bows and arrows!''

All but Tootles popped down their trees. He had a bow and arrow with
him, and Tink noted it, and rubbed her little hands.

``Quick, Tootles, quick,'' she screamed. ``Peter will be so pleased.''

Tootles excitedly fitted the arrow to his bow. ``Out of the way, Tink,''
he shouted, and then he fired, and Wendy fluttered to the ground with
an arrow in her breast.




Chapter VI.
THE LITTLE HOUSE


Foolish Tootles was standing like a conqueror over Wendy's body when
the other boys sprang, armed, from their trees.

``You are too late,'' he cried proudly, ``I have shot the Wendy. Peter
will be so pleased with me.''

Overhead Tinker Bell shouted ``Silly ass!'' and darted into hiding. The
others did not hear her. They had crowded round Wendy, and as they
looked a terrible silence fell upon the wood. If Wendy's heart had been
beating they would all have heard it.

Slightly was the first to speak. ``This is no bird,'' he said in a scared
voice. ``I think this must be a lady.''

``A lady?'' said Tootles, and fell a-trembling.

``And we have killed her,'' Nibs said hoarsely.

They all whipped off their caps.

``Now I see,'' Curly said: ``Peter was bringing her to us.'' He threw
himself sorrowfully on the ground.

``A lady to take care of us at last,'' said one of the twins, ``and you
have killed her!''

They were sorry for him, but sorrier for themselves, and when he took a
step nearer them they turned from him.

Tootles' face was very white, but there was a dignity about him now
that had never been there before.

``I did it,'' he said, reflecting. ``When ladies used to come to me in
dreams, I said, ‘Pretty mother, pretty mother.' But when at last she
really came, I shot her.''

He moved slowly away.

``Don't go,'' they called in pity.

``I must,'' he answered, shaking; ``I am so afraid of Peter.''

It was at this tragic moment that they heard a sound which made the
heart of every one of them rise to his mouth. They heard Peter crow.

``Peter!'' they cried, for it was always thus that he signalled his
return.

``Hide her,'' they whispered, and gathered hastily around Wendy. But
Tootles stood aloof.

Again came that ringing crow, and Peter dropped in front of them.
``Greetings, boys,'' he cried, and mechanically they saluted, and then
again was silence.

He frowned.

``I am back,'' he said hotly, ``why do you not cheer?''

They opened their mouths, but the cheers would not come. He overlooked
it in his haste to tell the glorious tidings.

``Great news, boys,'' he cried, ``I have brought at last a mother for you
all.''

Still no sound, except a little thud from Tootles as he dropped on his
knees.

``Have you not seen her?'' asked Peter, becoming troubled. ``She flew this
way.''

``Ah me!'' one voice said, and another said, ``Oh, mournful day.''

Tootles rose. ``Peter,'' he said quietly, ``I will show her to you,'' and
when the others would still have hidden her he said, ``Back, twins, let
Peter see.''

So they all stood back, and let him see, and after he had looked for a
little time he did not know what to do next.

``She is dead,'' he said uncomfortably. ``Perhaps she is frightened at
being dead.''

He thought of hopping off in a comic sort of way till he was out of
sight of her, and then never going near the spot any more. They would
all have been glad to follow if he had done this.

But there was the arrow. He took it from her heart and faced his band.

``Whose arrow?'' he demanded sternly.

``Mine, Peter,'' said Tootles on his knees.

``Oh, dastard hand,'' Peter said, and he raised the arrow to use it as a
dagger.

Tootles did not flinch. He bared his breast. ``Strike, Peter,'' he said
firmly, ``strike true.''

Twice did Peter raise the arrow, and twice did his hand fall. ``I cannot
strike,'' he said with awe, ``there is something stays my hand.''

All looked at him in wonder, save Nibs, who fortunately looked at
Wendy.

``It is she,'' he cried, ``the Wendy lady, see, her arm!''

Wonderful to relate, Wendy had raised her arm. Nibs bent over her and
listened reverently. ``I think she said, ‘Poor Tootles,''' he whispered.

``She lives,'' Peter said briefly.

Slightly cried instantly, ``The Wendy lady lives.''

Then Peter knelt beside her and found his button. You remember she had
put it on a chain that she wore round her neck.

``See,'' he said, ``the arrow struck against this. It is the kiss I gave
her. It has saved her life.''

``I remember kisses,'' Slightly interposed quickly, ``let me see it. Ay,
that's a kiss.''

Peter did not hear him. He was begging Wendy to get better quickly, so
that he could show her the mermaids. Of course she could not answer
yet, being still in a frightful faint; but from overhead came a wailing
note.

``Listen to Tink,'' said Curly, ``she is crying because the Wendy lives.''

Then they had to tell Peter of Tink's crime, and almost never had they
seen him look so stern.

``Listen, Tinker Bell,'' he cried, ``I am your friend no more. Begone from
me for ever.''

She flew on to his shoulder and pleaded, but he brushed her off. Not
until Wendy again raised her arm did he relent sufficiently to say,
``Well, not for ever, but for a whole week.''

Do you think Tinker Bell was grateful to Wendy for raising her arm? Oh
dear no, never wanted to pinch her so much. Fairies indeed are strange,
and Peter, who understood them best, often cuffed them.

But what to do with Wendy in her present delicate state of health?

``Let us carry her down into the house,'' Curly suggested.

``Ay,'' said Slightly, ``that is what one does with ladies.''

``No, no,'' Peter said, ``you must not touch her. It would not be
sufficiently respectful.''

``That,'' said Slightly, ``is what I was thinking.''

``But if she lies there,'' Tootles said, ``she will die.''

``Ay, she will die,'' Slightly admitted, ``but there is no way out.''

``Yes, there is,'' cried Peter. ``Let us build a little house round her.''

They were all delighted. ``Quick,'' he ordered them, ``bring me each of
you the best of what we have. Gut our house. Be sharp.''

In a moment they were as busy as tailors the night before a wedding.
They skurried this way and that, down for bedding, up for firewood, and
while they were at it, who should appear but John and Michael. As they
dragged along the ground they fell asleep standing, stopped, woke up,
moved another step and slept again.

``John, John,'' Michael would cry, ``wake up! Where is Nana, John, and
mother?''

And then John would rub his eyes and mutter, ``It is true, we did fly.''

You may be sure they were very relieved to find Peter.

``Hullo, Peter,'' they said.

``Hullo,'' replied Peter amicably, though he had quite forgotten them. He
was very busy at the moment measuring Wendy with his feet to see how
large a house she would need. Of course he meant to leave room for
chairs and a table. John and Michael watched him.

``Is Wendy asleep?'' they asked.

``Yes.''

``John,'' Michael proposed, ``let us wake her and get her to make supper
for us,'' but as he said it some of the other boys rushed on carrying
branches for the building of the house. ``Look at them!'' he cried.

``Curly,'' said Peter in his most captainy voice, ``see that these boys
help in the building of the house.''

``Ay, ay, sir.''

``Build a house?'' exclaimed John.

``For the Wendy,'' said Curly.

``For Wendy?'' John said, aghast. ``Why, she is only a girl!''

``That,'' explained Curly, ``is why we are her servants.''

``You? Wendy's servants!''

``Yes,'' said Peter, ``and you also. Away with them.''

The astounded brothers were dragged away to hack and hew and carry.
``Chairs and a fender first,'' Peter ordered. ``Then we shall build a
house round them.''

``Ay,'' said Slightly, ``that is how a house is built; it all comes back
to me.''

Peter thought of everything. ``Slightly,'' he cried, ``fetch a doctor.''

``Ay, ay,'' said Slightly at once, and disappeared, scratching his head.
But he knew Peter must be obeyed, and he returned in a moment, wearing
John's hat and looking solemn.

``Please, sir,'' said Peter, going to him, ``are you a doctor?''

The difference between him and the other boys at such a time was that
they knew it was make-believe, while to him make-believe and true were
exactly the same thing. This sometimes troubled them, as when they had
to make-believe that they had had their dinners.

If they broke down in their make-believe he rapped them on the
knuckles.

``Yes, my little man,'' Slightly anxiously replied, who had chapped
knuckles.

``Please, sir,'' Peter explained, ``a lady lies very ill.''

She was lying at their feet, but Slightly had the sense not to see her.

``Tut, tut, tut,'' he said, ``where does she lie?''

``In yonder glade.''

``I will put a glass thing in her mouth,'' said Slightly, and he
made-believe to do it, while Peter waited. It was an anxious moment
when the glass thing was withdrawn.

``How is she?'' inquired Peter.

``Tut, tut, tut,'' said Slightly, ``this has cured her.''

``I am glad!'' Peter cried.

``I will call again in the evening,'' Slightly said; ``give her beef tea
out of a cup with a spout to it;'' but after he had returned the hat to
John he blew big breaths, which was his habit on escaping from a
difficulty.

In the meantime the wood had been alive with the sound of axes; almost
everything needed for a cosy dwelling already lay at Wendy's feet.

``If only we knew,'' said one, ``the kind of house she likes best.''

``Peter,'' shouted another, ``she is moving in her sleep.''

``Her mouth opens,'' cried a third, looking respectfully into it. ``Oh,
lovely!''

``Perhaps she is going to sing in her sleep,'' said Peter. ``Wendy, sing
the kind of house you would like to have.''

Immediately, without opening her eyes, Wendy began to sing:

``I wish I had a pretty house,
    The littlest ever seen,
With funny little red walls
    And roof of mossy green.''

They gurgled with joy at this, for by the greatest good luck the
branches they had brought were sticky with red sap, and all the ground
was carpeted with moss. As they rattled up the little house they broke
into song themselves:

``We've built the little walls and roof
    And made a lovely door,
So tell us, mother Wendy,
    What are you wanting more?''

To this she answered greedily:

``Oh, really next I think I'll have
    Gay windows all about,
With roses peeping in, you know,
    And babies peeping out.''

With a blow of their fists they made windows, and large yellow leaves
were the blinds. But roses—?

``Roses,'' cried Peter sternly.

Quickly they made-believe to grow the loveliest roses up the walls.

Babies?

To prevent Peter ordering babies they hurried into song again:

``We've made the roses peeping out,
    The babes are at the door,
We cannot make ourselves, you know,
    'Cos we've been made before.''

Peter, seeing this to be a good idea, at once pretended that it was his
own. The house was quite beautiful, and no doubt Wendy was very cosy
within, though, of course, they could no longer see her. Peter strode
up and down, ordering finishing touches. Nothing escaped his eagle
eyes. Just when it seemed absolutely finished:

``There's no knocker on the door,'' he said.

They were very ashamed, but Tootles gave the sole of his shoe, and it
made an excellent knocker.

Absolutely finished now, they thought.

Not of bit of it. ``There's no chimney,'' Peter said; ``we must have a
chimney.''

``It certainly does need a chimney,'' said John importantly. This gave
Peter an idea. He snatched the hat off John's head, knocked out the
bottom, and put the hat on the roof. The little house was so pleased to
have such a capital chimney that, as if to say thank you, smoke
immediately began to come out of the hat.

Now really and truly it was finished. Nothing remained to do but to
knock.

``All look your best,'' Peter warned them; ``first impressions are awfully
important.''

He was glad no one asked him what first impressions are; they were all
too busy looking their best.

He knocked politely, and now the wood was as still as the children, not
a sound to be heard except from Tinker Bell, who was watching from a
branch and openly sneering.

What the boys were wondering was, would any one answer the knock? If a
lady, what would she be like?

The door opened and a lady came out. It was Wendy. They all whipped off
their hats.

She looked properly surprised, and this was just how they had hoped she
would look.

``Where am I?'' she said.

Of course Slightly was the first to get his word in. ``Wendy lady,'' he
said rapidly, ``for you we built this house.''

``Oh, say you're pleased,'' cried Nibs.

``Lovely, darling house,'' Wendy said, and they were the very words they
had hoped she would say.

``And we are your children,'' cried the twins.

Then all went on their knees, and holding out their arms cried, ``O
Wendy lady, be our mother.''

``Ought I?'' Wendy said, all shining. ``Of course it's frightfully
fascinating, but you see I am only a little girl. I have no real
experience.''

``That doesn't matter,'' said Peter, as if he were the only person
present who knew all about it, though he was really the one who knew
least. ``What we need is just a nice motherly person.''

``Oh dear!'' Wendy said, ``you see, I feel that is exactly what I am.''

``It is, it is,'' they all cried; ``we saw it at once.''

``Very well,'' she said, ``I will do my best. Come inside at once, you
naughty children; I am sure your feet are damp. And before I put you to
bed I have just time to finish the story of Cinderella.''

In they went; I don't know how there was room for them, but you can
squeeze very tight in the Neverland. And that was the first of the many
joyous evenings they had with Wendy. By and by she tucked them up in
the great bed in the home under the trees, but she herself slept that
night in the little house, and Peter kept watch outside with drawn
sword, for the pirates could be heard carousing far away and the wolves
were on the prowl. The little house looked so cosy and safe in the
darkness, with a bright light showing through its blinds, and the
chimney smoking beautifully, and Peter standing on guard. After a time
he fell asleep, and some unsteady fairies had to climb over him on
their way home from an orgy. Any of the other boys obstructing the
fairy path at night they would have mischiefed, but they just tweaked
Peter's nose and passed on.




Chapter VII.
THE HOME UNDER THE GROUND


One of the first things Peter did next day was to measure Wendy and
John and Michael for hollow trees. Hook, you remember, had sneered at
the boys for thinking they needed a tree apiece, but this was
ignorance, for unless your tree fitted you it was difficult to go up
and down, and no two of the boys were quite the same size. Once you
fitted, you drew in your breath at the top, and down you went at
exactly the right speed, while to ascend you drew in and let out
alternately, and so wriggled up. Of course, when you have mastered the
action you are able to do these things without thinking of them, and
nothing can be more graceful.

But you simply must fit, and Peter measures you for your tree as
carefully as for a suit of clothes: the only difference being that the
clothes are made to fit you, while you have to be made to fit the tree.
Usually it is done quite easily, as by your wearing too many garments
or too few, but if you are bumpy in awkward places or the only
available tree is an odd shape, Peter does some things to you, and
after that you fit. Once you fit, great care must be taken to go on
fitting, and this, as Wendy was to discover to her delight, keeps a
whole family in perfect condition.

Wendy and Michael fitted their trees at the first try, but John had to
be altered a little.

After a few days' practice they could go up and down as gaily as
buckets in a well. And how ardently they grew to love their home under
the ground; especially Wendy. It consisted of one large room, as all
houses should do, with a floor in which you could dig if you wanted to
go fishing, and in this floor grew stout mushrooms of a charming
colour, which were used as stools. A Never tree tried hard to grow in
the centre of the room, but every morning they sawed the trunk through,
level with the floor. By tea-time it was always about two feet high,
and then they put a door on top of it, the whole thus becoming a table;
as soon as they cleared away, they sawed off the trunk again, and thus
there was more room to play. There was an enormous fireplace which was
in almost any part of the room where you cared to light it, and across
this Wendy stretched strings, made of fibre, from which she suspended
her washing. The bed was tilted against the wall by day, and let down
at 6:30, when it filled nearly half the room; and all the boys slept in
it, except Michael, lying like sardines in a tin. There was a strict
rule against turning round until one gave the signal, when all turned
at once. Michael should have used it also, but Wendy would have a baby,
and he was the littlest, and you know what women are, and the short and
long of it is that he was hung up in a basket.

It was rough and simple, and not unlike what baby bears would have made
of an underground house in the same circumstances. But there was one
recess in the wall, no larger than a bird-cage, which was the private
apartment of Tinker Bell. It could be shut off from the rest of the
house by a tiny curtain, which Tink, who was most fastidious, always
kept drawn when dressing or undressing. No woman, however large, could
have had a more exquisite boudoir and bed-chamber combined. The couch,
as she always called it, was a genuine Queen Mab, with club legs; and
she varied the bedspreads according to what fruit-blossom was in
season. Her mirror was a Puss-in-Boots, of which there are now only
three, unchipped, known to fairy dealers; the washstand was Pie-crust
and reversible, the chest of drawers an authentic Charming the Sixth,
and the carpet and rugs the best (the early) period of Margery and
Robin. There was a chandelier from Tiddlywinks for the look of the
thing, but of course she lit the residence herself. Tink was very
contemptuous of the rest of the house, as indeed was perhaps
inevitable, and her chamber, though beautiful, looked rather conceited,
having the appearance of a nose permanently turned up.

I suppose it was all especially entrancing to Wendy, because those
rampagious boys of hers gave her so much to do. Really there were whole
weeks when, except perhaps with a stocking in the evening, she was
never above ground. The cooking, I can tell you, kept her nose to the
pot, and even if there was nothing in it, even if there was no pot, she
had to keep watching that it came aboil just the same. You never
exactly knew whether there would be a real meal or just a make-believe,
it all depended upon Peter's whim: he could eat, really eat, if it was
part of a game, but he could not stodge just to feel stodgy, which is
what most children like better than anything else; the next best thing
being to talk about it. Make-believe was so real to him that during a
meal of it you could see him getting rounder. Of course it was trying,
but you simply had to follow his lead, and if you could prove to him
that you were getting loose for your tree he let you stodge.

Wendy's favourite time for sewing and darning was after they had all
gone to bed. Then, as she expressed it, she had a breathing time for
herself; and she occupied it in making new things for them, and putting
double pieces on the knees, for they were all most frightfully hard on
their knees.

When she sat down to a basketful of their stockings, every heel with a
hole in it, she would fling up her arms and exclaim, ``Oh dear, I am
sure I sometimes think spinsters are to be envied!''

Her face beamed when she exclaimed this.

You remember about her pet wolf. Well, it very soon discovered that she
had come to the island and it found her out, and they just ran into
each other's arms. After that it followed her about everywhere.

As time wore on did she think much about the beloved parents she had
left behind her? This is a difficult question, because it is quite
impossible to say how time does wear on in the Neverland, where it is
calculated by moons and suns, and there are ever so many more of them
than on the mainland. But I am afraid that Wendy did not really worry
about her father and mother; she was absolutely confident that they
would always keep the window open for her to fly back by, and this gave
her complete ease of mind. What did disturb her at times was that John
remembered his parents vaguely only, as people he had once known, while
Michael was quite willing to believe that she was really his mother.
These things scared her a little, and nobly anxious to do her duty, she
tried to fix the old life in their minds by setting them examination
papers on it, as like as possible to the ones she used to do at school.
The other boys thought this awfully interesting, and insisted on
joining, and they made slates for themselves, and sat round the table,
writing and thinking hard about the questions she had written on
another slate and passed round. They were the most ordinary
questions—``What was the colour of Mother's eyes? Which was taller,
Father or Mother? Was Mother blonde or brunette? Answer all three
questions if possible.'' ``(A) Write an essay of not less than 40 words
on How I spent my last Holidays, or The Characters of Father and Mother
compared. Only one of these to be attempted.'' Or ``(1) Describe Mother's
laugh; (2) Describe Father's laugh; (3) Describe Mother's Party Dress;
(4) Describe the Kennel and its Inmate.''

They were just everyday questions like these, and when you could not
answer them you were told to make a cross; and it was really dreadful
what a number of crosses even John made. Of course the only boy who
replied to every question was Slightly, and no one could have been more
hopeful of coming out first, but his answers were perfectly ridiculous,
and he really came out last: a melancholy thing.

Peter did not compete. For one thing he despised all mothers except
Wendy, and for another he was the only boy on the island who could
neither write nor spell; not the smallest word. He was above all that
sort of thing.

By the way, the questions were all written in the past tense. What was
the colour of Mother's eyes, and so on. Wendy, you see, had been
forgetting, too.

Adventures, of course, as we shall see, were of daily occurrence; but
about this time Peter invented, with Wendy's help, a new game that
fascinated him enormously, until he suddenly had no more interest in
it, which, as you have been told, was what always happened with his
games. It consisted in pretending not to have adventures, in doing the
sort of thing John and Michael had been doing all their lives, sitting
on stools flinging balls in the air, pushing each other, going out for
walks and coming back without having killed so much as a grizzly. To
see Peter doing nothing on a stool was a great sight; he could not help
looking solemn at such times, to sit still seemed to him such a comic
thing to do. He boasted that he had gone walking for the good of his
health. For several suns these were the most novel of all adventures to
him; and John and Michael had to pretend to be delighted also;
otherwise he would have treated them severely.

He often went out alone, and when he came back you were never
absolutely certain whether he had had an adventure or not. He might
have forgotten it so completely that he said nothing about it; and then
when you went out you found the body; and, on the other hand, he might
say a great deal about it, and yet you could not find the body.
Sometimes he came home with his head bandaged, and then Wendy cooed
over him and bathed it in lukewarm water, while he told a dazzling
tale. But she was never quite sure, you know. There were, however, many
adventures which she knew to be true because she was in them herself,
and there were still more that were at least partly true, for the other
boys were in them and said they were wholly true. To describe them all
would require a book as large as an English-Latin, Latin-English
Dictionary, and the most we can do is to give one as a specimen of an
average hour on the island. The difficulty is which one to choose.
Should we take the brush with the redskins at Slightly Gulch? It was a
sanguinary affair, and especially interesting as showing one of Peter's
peculiarities, which was that in the middle of a fight he would
suddenly change sides. At the Gulch, when victory was still in the
balance, sometimes leaning this way and sometimes that, he called out,
``I'm redskin to-day; what are you, Tootles?'' And Tootles answered,
``Redskin; what are you, Nibs?'' and Nibs said, ``Redskin; what are you
Twin?'' and so on; and they were all redskins; and of course this would
have ended the fight had not the real redskins fascinated by Peter's
methods, agreed to be lost boys for that once, and so at it they all
went again, more fiercely than ever.

The extraordinary upshot of this adventure was—but we have not decided
yet that this is the adventure we are to narrate. Perhaps a better one
would be the night attack by the redskins on the house under the
ground, when several of them stuck in the hollow trees and had to be
pulled out like corks. Or we might tell how Peter saved Tiger Lily's
life in the Mermaids' Lagoon, and so made her his ally.

Or we could tell of that cake the pirates cooked so that the boys might
eat it and perish; and how they placed it in one cunning spot after
another; but always Wendy snatched it from the hands of her children,
so that in time it lost its succulence, and became as hard as a stone,
and was used as a missile, and Hook fell over it in the dark.

Or suppose we tell of the birds that were Peter's friends, particularly
of the Never bird that built in a tree overhanging the lagoon, and how
the nest fell into the water, and still the bird sat on her eggs, and
Peter gave orders that she was not to be disturbed. That is a pretty
story, and the end shows how grateful a bird can be; but if we tell it
we must also tell the whole adventure of the lagoon, which would of
course be telling two adventures rather than just one. A shorter
adventure, and quite as exciting, was Tinker Bell's attempt, with the
help of some street fairies, to have the sleeping Wendy conveyed on a
great floating leaf to the mainland. Fortunately the leaf gave way and
Wendy woke, thinking it was bath-time, and swam back. Or again, we
might choose Peter's defiance of the lions, when he drew a circle round
him on the ground with an arrow and dared them to cross it; and though
he waited for hours, with the other boys and Wendy looking on
breathlessly from trees, not one of them dared to accept his challenge.

Which of these adventures shall we choose? The best way will be to toss
for it.

I have tossed, and the lagoon has won. This almost makes one wish that
the gulch or the cake or Tink's leaf had won. Of course I could do it
again, and make it best out of three; however, perhaps fairest to stick
to the lagoon.




Chapter VIII.
THE MERMAIDS' LAGOON


If you shut your eyes and are a lucky one, you may see at times a
shapeless pool of lovely pale colours suspended in the darkness; then
if you squeeze your eyes tighter, the pool begins to take shape, and
the colours become so vivid that with another squeeze they must go on
fire. But just before they go on fire you see the lagoon. This is the
nearest you ever get to it on the mainland, just one heavenly moment;
if there could be two moments you might see the surf and hear the
mermaids singing.

The children often spent long summer days on this lagoon, swimming or
floating most of the time, playing the mermaid games in the water, and
so forth. You must not think from this that the mermaids were on
friendly terms with them: on the contrary, it was among Wendy's lasting
regrets that all the time she was on the island she never had a civil
word from one of them. When she stole softly to the edge of the lagoon
she might see them by the score, especially on Marooners' Rock, where
they loved to bask, combing out their hair in a lazy way that quite
irritated her; or she might even swim, on tiptoe as it were, to within
a yard of them, but then they saw her and dived, probably splashing her
with their tails, not by accident, but intentionally.

They treated all the boys in the same way, except of course Peter, who
chatted with them on Marooners' Rock by the hour, and sat on their
tails when they got cheeky. He gave Wendy one of their combs.

The most haunting time at which to see them is at the turn of the moon,
when they utter strange wailing cries; but the lagoon is dangerous for
mortals then, and until the evening of which we have now to tell, Wendy
had never seen the lagoon by moonlight, less from fear, for of course
Peter would have accompanied her, than because she had strict rules
about every one being in bed by seven. She was often at the lagoon,
however, on sunny days after rain, when the mermaids come up in
extraordinary numbers to play with their bubbles. The bubbles of many
colours made in rainbow water they treat as balls, hitting them gaily
from one to another with their tails, and trying to keep them in the
rainbow till they burst. The goals are at each end of the rainbow, and
the keepers only are allowed to use their hands. Sometimes a dozen of
these games will be going on in the lagoon at a time, and it is quite a
pretty sight.

But the moment the children tried to join in they had to play by
themselves, for the mermaids immediately disappeared. Nevertheless we
have proof that they secretly watched the interlopers, and were not
above taking an idea from them; for John introduced a new way of
hitting the bubble, with the head instead of the hand, and the mermaids
adopted it. This is the one mark that John has left on the Neverland.

It must also have been rather pretty to see the children resting on a
rock for half an hour after their mid-day meal. Wendy insisted on their
doing this, and it had to be a real rest even though the meal was
make-believe. So they lay there in the sun, and their bodies glistened
in it, while she sat beside them and looked important.

It was one such day, and they were all on Marooners' Rock. The rock was
not much larger than their great bed, but of course they all knew how
not to take up much room, and they were dozing, or at least lying with
their eyes shut, and pinching occasionally when they thought Wendy was
not looking. She was very busy, stitching.

While she stitched a change came to the lagoon. Little shivers ran over
it, and the sun went away and shadows stole across the water, turning
it cold. Wendy could no longer see to thread her needle, and when she
looked up, the lagoon that had always hitherto been such a laughing
place seemed formidable and unfriendly.

It was not, she knew, that night had come, but something as dark as
night had come. No, worse than that. It had not come, but it had sent
that shiver through the sea to say that it was coming. What was it?

There crowded upon her all the stories she had been told of Marooners'
Rock, so called because evil captains put sailors on it and leave them
there to drown. They drown when the tide rises, for then it is
submerged.

Of course she should have roused the children at once; not merely
because of the unknown that was stalking toward them, but because it
was no longer good for them to sleep on a rock grown chilly. But she
was a young mother and she did not know this; she thought you simply
must stick to your rule about half an hour after the mid-day meal. So,
though fear was upon her, and she longed to hear male voices, she would
not waken them. Even when she heard the sound of muffled oars, though
her heart was in her mouth, she did not waken them. She stood over them
to let them have their sleep out. Was it not brave of Wendy?

It was well for those boys then that there was one among them who could
sniff danger even in his sleep. Peter sprang erect, as wide awake at
once as a dog, and with one warning cry he roused the others.

He stood motionless, one hand to his ear.

``Pirates!'' he cried. The others came closer to him. A strange smile was
playing about his face, and Wendy saw it and shuddered. While that
smile was on his face no one dared address him; all they could do was
to stand ready to obey. The order came sharp and incisive.

``Dive!''

There was a gleam of legs, and instantly the lagoon seemed deserted.
Marooners' Rock stood alone in the forbidding waters as if it were
itself marooned.

The boat drew nearer. It was the pirate dinghy, with three figures in
her, Smee and Starkey, and the third a captive, no other than Tiger
Lily. Her hands and ankles were tied, and she knew what was to be her
fate. She was to be left on the rock to perish, an end to one of her
race more terrible than death by fire or torture, for is it not written
in the book of the tribe that there is no path through water to the
happy hunting-ground? Yet her face was impassive; she was the daughter
of a chief, she must die as a chief's daughter, it is enough.

They had caught her boarding the pirate ship with a knife in her mouth.
No watch was kept on the ship, it being Hook's boast that the wind of
his name guarded the ship for a mile around. Now her fate would help to
guard it also. One more wail would go the round in that wind by night.

In the gloom that they brought with them the two pirates did not see
the rock till they crashed into it.

``Luff, you lubber,'' cried an Irish voice that was Smee's; ``here's the
rock. Now, then, what we have to do is to hoist the redskin on to it
and leave her here to drown.''

It was the work of one brutal moment to land the beautiful girl on the
rock; she was too proud to offer a vain resistance.

Quite near the rock, but out of sight, two heads were bobbing up and
down, Peter's and Wendy's. Wendy was crying, for it was the first
tragedy she had seen. Peter had seen many tragedies, but he had
forgotten them all. He was less sorry than Wendy for Tiger Lily: it was
two against one that angered him, and he meant to save her. An easy way
would have been to wait until the pirates had gone, but he was never
one to choose the easy way.

There was almost nothing he could not do, and he now imitated the voice
of Hook.

``Ahoy there, you lubbers!'' he called. It was a marvellous imitation.

``The captain!'' said the pirates, staring at each other in surprise.

``He must be swimming out to us,'' Starkey said, when they had looked for
him in vain.

``We are putting the redskin on the rock,'' Smee called out.

``Set her free,'' came the astonishing answer.

``Free!''

``Yes, cut her bonds and let her go.''

``But, captain—''

``At once, d'ye hear,'' cried Peter, ``or I'll plunge my hook in you.''

``This is queer!'' Smee gasped.

``Better do what the captain orders,'' said Starkey nervously.

``Ay, ay,'' Smee said, and he cut Tiger Lily's cords. At once like an eel
she slid between Starkey's legs into the water.

Of course Wendy was very elated over Peter's cleverness; but she knew
that he would be elated also and very likely crow and thus betray
himself, so at once her hand went out to cover his mouth. But it was
stayed even in the act, for ``Boat ahoy!'' rang over the lagoon in Hook's
voice, and this time it was not Peter who had spoken.

Peter may have been about to crow, but his face puckered in a whistle
of surprise instead.

``Boat ahoy!'' again came the voice.

Now Wendy understood. The real Hook was also in the water.

He was swimming to the boat, and as his men showed a light to guide him
he had soon reached them. In the light of the lantern Wendy saw his
hook grip the boat's side; she saw his evil swarthy face as he rose
dripping from the water, and, quaking, she would have liked to swim
away, but Peter would not budge. He was tingling with life and also
top-heavy with conceit. ``Am I not a wonder, oh, I am a wonder!'' he
whispered to her, and though she thought so also, she was really glad
for the sake of his reputation that no one heard him except herself.

He signed to her to listen.

The two pirates were very curious to know what had brought their
captain to them, but he sat with his head on his hook in a position of
profound melancholy.

``Captain, is all well?'' they asked timidly, but he answered with a
hollow moan.

``He sighs,'' said Smee.

``He sighs again,'' said Starkey.

``And yet a third time he sighs,'' said Smee.

Then at last he spoke passionately.

``The game's up,'' he cried, ``those boys have found a mother.''

Affrighted though she was, Wendy swelled with pride.

``O evil day!'' cried Starkey.

``What's a mother?'' asked the ignorant Smee.

Wendy was so shocked that she exclaimed. ``He doesn't know!'' and always
after this she felt that if you could have a pet pirate Smee would be
her one.

Peter pulled her beneath the water, for Hook had started up, crying,
``What was that?''

``I heard nothing,'' said Starkey, raising the lantern over the waters,
and as the pirates looked they saw a strange sight. It was the nest I
have told you of, floating on the lagoon, and the Never bird was
sitting on it.

``See,'' said Hook in answer to Smee's question, ``that is a mother. What
a lesson! The nest must have fallen into the water, but would the
mother desert her eggs? No.''

There was a break in his voice, as if for a moment he recalled innocent
days when—but he brushed away this weakness with his hook.

Smee, much impressed, gazed at the bird as the nest was borne past, but
the more suspicious Starkey said, ``If she is a mother, perhaps she is
hanging about here to help Peter.''

Hook winced. ``Ay,'' he said, ``that is the fear that haunts me.''

He was roused from this dejection by Smee's eager voice.

``Captain,'' said Smee, ``could we not kidnap these boys' mother and make
her our mother?''

``It is a princely scheme,'' cried Hook, and at once it took practical
shape in his great brain. ``We will seize the children and carry them to
the boat: the boys we will make walk the plank, and Wendy shall be our
mother.''

Again Wendy forgot herself.

``Never!'' she cried, and bobbed.

``What was that?''

But they could see nothing. They thought it must have been a leaf in
the wind. ``Do you agree, my bullies?'' asked Hook.

``There is my hand on it,'' they both said.

``And there is my hook. Swear.''

They all swore. By this time they were on the rock, and suddenly Hook
remembered Tiger Lily.

``Where is the redskin?'' he demanded abruptly.

He had a playful humour at moments, and they thought this was one of
the moments.

``That is all right, captain,'' Smee answered complacently; ``we let her
go.''

``Let her go!'' cried Hook.

``'Twas your own orders,'' the bo'sun faltered.

``You called over the water to us to let her go,'' said Starkey.

``Brimstone and gall,'' thundered Hook, ``what cozening is going on here!''
His face had gone black with rage, but he saw that they believed their
words, and he was startled. ``Lads,'' he said, shaking a little, ``I gave
no such order.''

``It is passing queer,'' Smee said, and they all fidgeted uncomfortably.
Hook raised his voice, but there was a quiver in it.

``Spirit that haunts this dark lagoon to-night,'' he cried, ``dost hear
me?''

Of course Peter should have kept quiet, but of course he did not. He
immediately answered in Hook's voice:

``Odds, bobs, hammer and tongs, I hear you.''

In that supreme moment Hook did not blanch, even at the gills, but Smee
and Starkey clung to each other in terror.

``Who are you, stranger? Speak!'' Hook demanded.

``I am James Hook,'' replied the voice, ``captain of the _Jolly Roger_.''

``You are not; you are not,'' Hook cried hoarsely.

``Brimstone and gall,'' the voice retorted, ``say that again, and I'll
cast anchor in you.''

Hook tried a more ingratiating manner. ``If you are Hook,'' he said
almost humbly, ``come tell me, who am I?''

``A codfish,'' replied the voice, ``only a codfish.''

``A codfish!'' Hook echoed blankly, and it was then, but not till then,
that his proud spirit broke. He saw his men draw back from him.

``Have we been captained all this time by a codfish!'' they muttered. ``It
is lowering to our pride.''

They were his dogs snapping at him, but, tragic figure though he had
become, he scarcely heeded them. Against such fearful evidence it was
not their belief in him that he needed, it was his own. He felt his ego
slipping from him. ``Don't desert me, bully,'' he whispered hoarsely to
it.

In his dark nature there was a touch of the feminine, as in all the
great pirates, and it sometimes gave him intuitions. Suddenly he tried
the guessing game.

``Hook,'' he called, ``have you another voice?''

Now Peter could never resist a game, and he answered blithely in his
own voice, ``I have.''

``And another name?''

``Ay, ay.''

``Vegetable?'' asked Hook.

``No.''

``Mineral?''

``No.''

``Animal?''

``Yes.''

``Man?''

``No!'' This answer rang out scornfully.

``Boy?''

``Yes.''

``Ordinary boy?''

``No!''

``Wonderful boy?''

To Wendy's pain the answer that rang out this time was ``Yes.''

``Are you in England?''

``No.''

``Are you here?''

``Yes.''

Hook was completely puzzled. ``You ask him some questions,'' he said to
the others, wiping his damp brow.

Smee reflected. ``I can't think of a thing,'' he said regretfully.

``Can't guess, can't guess!'' crowed Peter. ``Do you give it up?''

Of course in his pride he was carrying the game too far, and the
miscreants saw their chance.

``Yes, yes,'' they answered eagerly.

``Well, then,'' he cried, ``I am Peter Pan.''

Pan!

In a moment Hook was himself again, and Smee and Starkey were his
faithful henchmen.

``Now we have him,'' Hook shouted. ``Into the water, Smee. Starkey, mind
the boat. Take him dead or alive!''

He leaped as he spoke, and simultaneously came the gay voice of Peter.

``Are you ready, boys?''

``Ay, ay,'' from various parts of the lagoon.

``Then lam into the pirates.''

The fight was short and sharp. First to draw blood was John, who
gallantly climbed into the boat and held Starkey. There was fierce
struggle, in which the cutlass was torn from the pirate's grasp. He
wriggled overboard and John leapt after him. The dinghy drifted away.

Here and there a head bobbed up in the water, and there was a flash of
steel followed by a cry or a whoop. In the confusion some struck at
their own side. The corkscrew of Smee got Tootles in the fourth rib,
but he was himself pinked in turn by Curly. Farther from the rock
Starkey was pressing Slightly and the twins hard.

Where all this time was Peter? He was seeking bigger game.

The others were all brave boys, and they must not be blamed for backing
from the pirate captain. His iron claw made a circle of dead water
round him, from which they fled like affrighted fishes.

But there was one who did not fear him: there was one prepared to enter
that circle.

Strangely, it was not in the water that they met. Hook rose to the rock
to breathe, and at the same moment Peter scaled it on the opposite
side. The rock was slippery as a ball, and they had to crawl rather
than climb. Neither knew that the other was coming. Each feeling for a
grip met the other's arm: in surprise they raised their heads; their
faces were almost touching; so they met.

Some of the greatest heroes have confessed that just before they fell
to they had a sinking. Had it been so with Peter at that moment I would
admit it. After all, he was the only man that the Sea-Cook had feared.
But Peter had no sinking, he had one feeling only, gladness; and he
gnashed his pretty teeth with joy. Quick as thought he snatched a knife
from Hook's belt and was about to drive it home, when he saw that he
was higher up the rock than his foe. It would not have been fighting
fair. He gave the pirate a hand to help him up.

It was then that Hook bit him.

Not the pain of this but its unfairness was what dazed Peter. It made
him quite helpless. He could only stare, horrified. Every child is
affected thus the first time he is treated unfairly. All he thinks he
has a right to when he comes to you to be yours is fairness. After you
have been unfair to him he will love you again, but will never
afterwards be quite the same boy. No one ever gets over the first
unfairness; no one except Peter. He often met it, but he always forgot
it. I suppose that was the real difference between him and all the
rest.

So when he met it now it was like the first time; and he could just
stare, helpless. Twice the iron hand clawed him.

A few moments afterwards the other boys saw Hook in the water striking
wildly for the ship; no elation on the pestilent face now, only white
fear, for the crocodile was in dogged pursuit of him. On ordinary
occasions the boys would have swum alongside cheering; but now they
were uneasy, for they had lost both Peter and Wendy, and were scouring
the lagoon for them, calling them by name. They found the dinghy and
went home in it, shouting ``Peter, Wendy'' as they went, but no answer
came save mocking laughter from the mermaids. ``They must be swimming
back or flying,'' the boys concluded. They were not very anxious,
because they had such faith in Peter. They chuckled, boylike, because
they would be late for bed; and it was all mother Wendy's fault!

When their voices died away there came cold silence over the lagoon,
and then a feeble cry.

``Help, help!''

Two small figures were beating against the rock; the girl had fainted
and lay on the boy's arm. With a last effort Peter pulled her up the
rock and then lay down beside her. Even as he also fainted he saw that
the water was rising. He knew that they would soon be drowned, but he
could do no more.

As they lay side by side a mermaid caught Wendy by the feet, and began
pulling her softly into the water. Peter, feeling her slip from him,
woke with a start, and was just in time to draw her back. But he had to
tell her the truth.

``We are on the rock, Wendy,'' he said, ``but it is growing smaller. Soon
the water will be over it.''

She did not understand even now.

``We must go,'' she said, almost brightly.

``Yes,'' he answered faintly.

``Shall we swim or fly, Peter?''

He had to tell her.

``Do you think you could swim or fly as far as the island, Wendy,
without my help?''

She had to admit that she was too tired.

He moaned.

``What is it?'' she asked, anxious about him at once.

``I can't help you, Wendy. Hook wounded me. I can neither fly nor swim.''

``Do you mean we shall both be drowned?''

``Look how the water is rising.''

They put their hands over their eyes to shut out the sight. They
thought they would soon be no more. As they sat thus something brushed
against Peter as light as a kiss, and stayed there, as if saying
timidly, ``Can I be of any use?''

It was the tail of a kite, which Michael had made some days before. It
had torn itself out of his hand and floated away.

``Michael's kite,'' Peter said without interest, but next moment he had
seized the tail, and was pulling the kite toward him.

``It lifted Michael off the ground,'' he cried; ``why should it not carry
you?''

``Both of us!''

``It can't lift two; Michael and Curly tried.''

``Let us draw lots,'' Wendy said bravely.

``And you a lady; never.'' Already he had tied the tail round her. She
clung to him; she refused to go without him; but with a ``Good-bye,
Wendy,'' he pushed her from the rock; and in a few minutes she was borne
out of his sight. Peter was alone on the lagoon.

The rock was very small now; soon it would be submerged. Pale rays of
light tiptoed across the waters; and by and by there was to be heard a
sound at once the most musical and the most melancholy in the world:
the mermaids calling to the moon.

Peter was not quite like other boys; but he was afraid at last. A
tremour ran through him, like a shudder passing over the sea; but on
the sea one shudder follows another till there are hundreds of them,
and Peter felt just the one. Next moment he was standing erect on the
rock again, with that smile on his face and a drum beating within him.
It was saying, ``To die will be an awfully big adventure.''




Chapter IX.
THE NEVER BIRD


The last sound Peter heard before he was quite alone were the mermaids
retiring one by one to their bedchambers under the sea. He was too far
away to hear their doors shut; but every door in the coral caves where
they live rings a tiny bell when it opens or closes (as in all the
nicest houses on the mainland), and he heard the bells.

Steadily the waters rose till they were nibbling at his feet; and to
pass the time until they made their final gulp, he watched the only
thing on the lagoon. He thought it was a piece of floating paper,
perhaps part of the kite, and wondered idly how long it would take to
drift ashore.

Presently he noticed as an odd thing that it was undoubtedly out upon
the lagoon with some definite purpose, for it was fighting the tide,
and sometimes winning; and when it won, Peter, always sympathetic to
the weaker side, could not help clapping; it was such a gallant piece
of paper.

It was not really a piece of paper; it was the Never bird, making
desperate efforts to reach Peter on the nest. By working her wings, in
a way she had learned since the nest fell into the water, she was able
to some extent to guide her strange craft, but by the time Peter
recognised her she was very exhausted. She had come to save him, to
give him her nest, though there were eggs in it. I rather wonder at the
bird, for though he had been nice to her, he had also sometimes
tormented her. I can suppose only that, like Mrs. Darling and the rest
of them, she was melted because he had all his first teeth.

She called out to him what she had come for, and he called out to her
what she was doing there; but of course neither of them understood the
other's language. In fanciful stories people can talk to the birds
freely, and I wish for the moment I could pretend that this were such a
story, and say that Peter replied intelligently to the Never bird; but
truth is best, and I want to tell you only what really happened. Well,
not only could they not understand each other, but they forgot their
manners.

``I—want—you—to—get—into—the—nest,'' the bird called, speaking as slowly
and distinctly as possible, ``and—then—you—can—drift—ashore,
but—I—am—too—tired—to—bring—it—any—nearer—so—you—must—try
to—swim—to—it.''

``What are you quacking about?'' Peter answered. ``Why don't you let the
nest drift as usual?''

``I—want—you—'' the bird said, and repeated it all over.

Then Peter tried slow and distinct.

``What—are—you—quacking—about?'' and so on.

The Never bird became irritated; they have very short tempers.

``You dunderheaded little jay!'' she screamed, ``Why don't you do as I
tell you?''

Peter felt that she was calling him names, and at a venture he retorted
hotly:

``So are you!''

Then rather curiously they both snapped out the same remark:

``Shut up!''

``Shut up!''

Nevertheless the bird was determined to save him if she could, and by
one last mighty effort she propelled the nest against the rock. Then up
she flew; deserting her eggs, so as to make her meaning clear.

Then at last he understood, and clutched the nest and waved his thanks
to the bird as she fluttered overhead. It was not to receive his
thanks, however, that she hung there in the sky; it was not even to
watch him get into the nest; it was to see what he did with her eggs.

There were two large white eggs, and Peter lifted them up and
reflected. The bird covered her face with her wings, so as not to see
the last of them; but she could not help peeping between the feathers.

I forget whether I have told you that there was a stave on the rock,
driven into it by some buccaneers of long ago to mark the site of
buried treasure. The children had discovered the glittering hoard, and
when in a mischievous mood used to fling showers of moidores, diamonds,
pearls and pieces of eight to the gulls, who pounced upon them for
food, and then flew away, raging at the scurvy trick that had been
played upon them. The stave was still there, and on it Starkey had hung
his hat, a deep tarpaulin, watertight, with a broad brim. Peter put the
eggs into this hat and set it on the lagoon. It floated beautifully.

The Never bird saw at once what he was up to, and screamed her
admiration of him; and, alas, Peter crowed his agreement with her. Then
he got into the nest, reared the stave in it as a mast, and hung up his
shirt for a sail. At the same moment the bird fluttered down upon the
hat and once more sat snugly on her eggs. She drifted in one direction,
and he was borne off in another, both cheering.

Of course when Peter landed he beached his barque in a place where the
bird would easily find it; but the hat was such a great success that
she abandoned the nest. It drifted about till it went to pieces, and
often Starkey came to the shore of the lagoon, and with many bitter
feelings watched the bird sitting on his hat. As we shall not see her
again, it may be worth mentioning here that all Never birds now build
in that shape of nest, with a broad brim on which the youngsters take
an airing.

Great were the rejoicings when Peter reached the home under the ground
almost as soon as Wendy, who had been carried hither and thither by the
kite. Every boy had adventures to tell; but perhaps the biggest
adventure of all was that they were several hours late for bed. This so
inflated them that they did various dodgy things to get staying up
still longer, such as demanding bandages; but Wendy, though glorying in
having them all home again safe and sound, was scandalised by the
lateness of the hour, and cried, ``To bed, to bed,'' in a voice that had
to be obeyed. Next day, however, she was awfully tender, and gave out
bandages to every one, and they played till bed-time at limping about
and carrying their arms in slings.




Chapter X.
THE HAPPY HOME


One important result of the brush on the lagoon was that it made the
redskins their friends. Peter had saved Tiger Lily from a dreadful
fate, and now there was nothing she and her braves would not do for
him. All night they sat above, keeping watch over the home under the
ground and awaiting the big attack by the pirates which obviously could
not be much longer delayed. Even by day they hung about, smoking the
pipe of peace, and looking almost as if they wanted tit-bits to eat.

They called Peter the Great White Father, prostrating themselves before
him; and he liked this tremendously, so that it was not really good for
him.

``The great white father,'' he would say to them in a very lordly manner,
as they grovelled at his feet, ``is glad to see the Piccaninny warriors
protecting his wigwam from the pirates.''

``Me Tiger Lily,'' that lovely creature would reply. ``Peter Pan save me,
me his velly nice friend. Me no let pirates hurt him.''

She was far too pretty to cringe in this way, but Peter thought it his
due, and he would answer condescendingly, ``It is good. Peter Pan has
spoken.''

Always when he said, ``Peter Pan has spoken,'' it meant that they must
now shut up, and they accepted it humbly in that spirit; but they were
by no means so respectful to the other boys, whom they looked upon as
just ordinary braves. They said ``How-do?'' to them, and things like
that; and what annoyed the boys was that Peter seemed to think this all
right.

Secretly Wendy sympathised with them a little, but she was far too
loyal a housewife to listen to any complaints against father. ``Father
knows best,'' she always said, whatever her private opinion must be. Her
private opinion was that the redskins should not call her a squaw.

We have now reached the evening that was to be known among them as the
Night of Nights, because of its adventures and their upshot. The day,
as if quietly gathering its forces, had been almost uneventful, and now
the redskins in their blankets were at their posts above, while, below,
the children were having their evening meal; all except Peter, who had
gone out to get the time. The way you got the time on the island was to
find the crocodile, and then stay near him till the clock struck.

The meal happened to be a make-believe tea, and they sat around the
board, guzzling in their greed; and really, what with their chatter and
recriminations, the noise, as Wendy said, was positively deafening. To
be sure, she did not mind noise, but she simply would not have them
grabbing things, and then excusing themselves by saying that Tootles
had pushed their elbow. There was a fixed rule that they must never hit
back at meals, but should refer the matter of dispute to Wendy by
raising the right arm politely and saying, ``I complain of so-and-so;''
but what usually happened was that they forgot to do this or did it too
much.

``Silence,'' cried Wendy when for the twentieth time she had told them
that they were not all to speak at once. ``Is your mug empty, Slightly
darling?''

``Not quite empty, mummy,'' Slightly said, after looking into an
imaginary mug.

``He hasn't even begun to drink his milk,'' Nibs interposed.

This was telling, and Slightly seized his chance.

``I complain of Nibs,'' he cried promptly.

John, however, had held up his hand first.

``Well, John?''

``May I sit in Peter's chair, as he is not here?''

``Sit in father's chair, John!'' Wendy was scandalised. ``Certainly not.''

``He is not really our father,'' John answered. ``He didn't even know how
a father does till I showed him.''

This was grumbling. ``We complain of John,'' cried the twins.

Tootles held up his hand. He was so much the humblest of them, indeed
he was the only humble one, that Wendy was specially gentle with him.

``I don't suppose,'' Tootles said diffidently, ``that I could be father.''

``No, Tootles.''

Once Tootles began, which was not very often, he had a silly way of
going on.

``As I can't be father,'' he said heavily, ``I don't suppose, Michael, you
would let me be baby?''

``No, I won't,'' Michael rapped out. He was already in his basket.

``As I can't be baby,'' Tootles said, getting heavier and heavier and
heavier, ``do you think I could be a twin?''

``No, indeed,'' replied the twins; ``it's awfully difficult to be a twin.''

``As I can't be anything important,'' said Tootles, ``would any of you
like to see me do a trick?''

``No,'' they all replied.

Then at last he stopped. ``I hadn't really any hope,'' he said.

The hateful telling broke out again.

``Slightly is coughing on the table.''

``The twins began with cheese-cakes.''

``Curly is taking both butter and honey.''

``Nibs is speaking with his mouth full.''

``I complain of the twins.''

``I complain of Curly.''

``I complain of Nibs.''

``Oh dear, oh dear,'' cried Wendy, ``I'm sure I sometimes think that
spinsters are to be envied.''

She told them to clear away, and sat down to her work-basket, a heavy
load of stockings and every knee with a hole in it as usual.

``Wendy,'' remonstrated Michael, ``I'm too big for a cradle.''

``I must have somebody in a cradle,'' she said almost tartly, ``and you
are the littlest. A cradle is such a nice homely thing to have about a
house.''

While she sewed they played around her; such a group of happy faces and
dancing limbs lit up by that romantic fire. It had become a very
familiar scene, this, in the home under the ground, but we are looking
on it for the last time.

There was a step above, and Wendy, you may be sure, was the first to
recognize it.

``Children, I hear your father's step. He likes you to meet him at the
door.''

Above, the redskins crouched before Peter.

``Watch well, braves. I have spoken.''

And then, as so often before, the gay children dragged him from his
tree. As so often before, but never again.

He had brought nuts for the boys as well as the correct time for Wendy.

``Peter, you just spoil them, you know,'' Wendy simpered.

``Ah, old lady,'' said Peter, hanging up his gun.

``It was me told him mothers are called old lady,'' Michael whispered to
Curly.

``I complain of Michael,'' said Curly instantly.

The first twin came to Peter. ``Father, we want to dance.''

``Dance away, my little man,'' said Peter, who was in high good humour.

``But we want you to dance.''

Peter was really the best dancer among them, but he pretended to be
scandalised.

``Me! My old bones would rattle!''

``And mummy too.''

``What,'' cried Wendy, ``the mother of such an armful, dance!''

``But on a Saturday night,'' Slightly insinuated.

It was not really Saturday night, at least it may have been, for they
had long lost count of the days; but always if they wanted to do
anything special they said this was Saturday night, and then they did
it.

``Of course it is Saturday night, Peter,'' Wendy said, relenting.

``People of our figure, Wendy!''

``But it is only among our own progeny.''

``True, true.''

So they were told they could dance, but they must put on their nighties
first.

``Ah, old lady,'' Peter said aside to Wendy, warming himself by the fire
and looking down at her as she sat turning a heel, ``there is nothing
more pleasant of an evening for you and me when the day's toil is over
than to rest by the fire with the little ones near by.''

``It is sweet, Peter, isn't it?'' Wendy said, frightfully gratified.
``Peter, I think Curly has your nose.''

``Michael takes after you.''

She went to him and put her hand on his shoulder.

``Dear Peter,'' she said, ``with such a large family, of course, I have
now passed my best, but you don't want to change me, do you?''

``No, Wendy.''

Certainly he did not want a change, but he looked at her uncomfortably,
blinking, you know, like one not sure whether he was awake or asleep.

``Peter, what is it?''

``I was just thinking,'' he said, a little scared. ``It is only
make-believe, isn't it, that I am their father?''

``Oh yes,'' Wendy said primly.

``You see,'' he continued apologetically, ``it would make me seem so old
to be their real father.''

``But they are ours, Peter, yours and mine.''

``But not really, Wendy?'' he asked anxiously.

``Not if you don't wish it,'' she replied; and she distinctly heard his
sigh of relief. ``Peter,'' she asked, trying to speak firmly, ``what are
your exact feelings to me?''

``Those of a devoted son, Wendy.''

``I thought so,'' she said, and went and sat by herself at the extreme
end of the room.

``You are so queer,'' he said, frankly puzzled, ``and Tiger Lily is just
the same. There is something she wants to be to me, but she says it is
not my mother.''

``No, indeed, it is not,'' Wendy replied with frightful emphasis. Now we
know why she was prejudiced against the redskins.

``Then what is it?''

``It isn't for a lady to tell.''

``Oh, very well,'' Peter said, a little nettled. ``Perhaps Tinker Bell
will tell me.''

``Oh yes, Tinker Bell will tell you,'' Wendy retorted scornfully. ``She is
an abandoned little creature.''

Here Tink, who was in her bedroom, eavesdropping, squeaked out
something impudent.

``She says she glories in being abandoned,'' Peter interpreted.

He had a sudden idea. ``Perhaps Tink wants to be my mother?''

``You silly ass!'' cried Tinker Bell in a passion.

She had said it so often that Wendy needed no translation.

``I almost agree with her,'' Wendy snapped. Fancy Wendy snapping! But she
had been much tried, and she little knew what was to happen before the
night was out. If she had known she would not have snapped.

None of them knew. Perhaps it was best not to know. Their ignorance
gave them one more glad hour; and as it was to be their last hour on
the island, let us rejoice that there were sixty glad minutes in it.
They sang and danced in their night-gowns. Such a deliciously creepy
song it was, in which they pretended to be frightened at their own
shadows, little witting that so soon shadows would close in upon them,
from whom they would shrink in real fear. So uproariously gay was the
dance, and how they buffeted each other on the bed and out of it! It
was a pillow fight rather than a dance, and when it was finished, the
pillows insisted on one bout more, like partners who know that they may
never meet again. The stories they told, before it was time for Wendy's
good-night story! Even Slightly tried to tell a story that night, but
the beginning was so fearfully dull that it appalled not only the
others but himself, and he said gloomily:

``Yes, it is a dull beginning. I say, let us pretend that it is the
end.''

And then at last they all got into bed for Wendy's story, the story
they loved best, the story Peter hated. Usually when she began to tell
this story he left the room or put his hands over his ears; and
possibly if he had done either of those things this time they might all
still be on the island. But to-night he remained on his stool; and we
shall see what happened.




Chapter XI.
WENDY'S STORY


``Listen, then,'' said Wendy, settling down to her story, with Michael at
her feet and seven boys in the bed. ``There was once a gentleman—''

``I had rather he had been a lady,'' Curly said.

``I wish he had been a white rat,'' said Nibs.

``Quiet,'' their mother admonished them. ``There was a lady also, and—''

``Oh, mummy,'' cried the first twin, ``you mean that there is a lady also,
don't you? She is not dead, is she?''

``Oh, no.''

``I am awfully glad she isn't dead,'' said Tootles. ``Are you glad, John?''

``Of course I am.''

``Are you glad, Nibs?''

``Rather.''

``Are you glad, Twins?''

``We are glad.''

``Oh dear,'' sighed Wendy.

``Little less noise there,'' Peter called out, determined that she should
have fair play, however beastly a story it might be in his opinion.

``The gentleman's name,'' Wendy continued, ``was Mr. Darling, and her name
was Mrs. Darling.''

``I knew them,'' John said, to annoy the others.

``I think I knew them,'' said Michael rather doubtfully.

``They were married, you know,'' explained Wendy, ``and what do you think
they had?''

``White rats,'' cried Nibs, inspired.

``No.''

``It's awfully puzzling,'' said Tootles, who knew the story by heart.

``Quiet, Tootles. They had three descendants.''

``What is descendants?''

``Well, you are one, Twin.''

``Did you hear that, John? I am a descendant.''

``Descendants are only children,'' said John.

``Oh dear, oh dear,'' sighed Wendy. ``Now these three children had a
faithful nurse called Nana; but Mr. Darling was angry with her and
chained her up in the yard, and so all the children flew away.''

``It's an awfully good story,'' said Nibs.

``They flew away,'' Wendy continued, ``to the Neverland, where the lost
children are.''

``I just thought they did,'' Curly broke in excitedly. ``I don't know how
it is, but I just thought they did!''

``O Wendy,'' cried Tootles, ``was one of the lost children called
Tootles?''

``Yes, he was.''

``I am in a story. Hurrah, I am in a story, Nibs.''

``Hush. Now I want you to consider the feelings of the unhappy parents
with all their children flown away.''

``Oo!'' they all moaned, though they were not really considering the
feelings of the unhappy parents one jot.

``Think of the empty beds!''

``Oo!''

``It's awfully sad,'' the first twin said cheerfully.

``I don't see how it can have a happy ending,'' said the second twin. ``Do
you, Nibs?''

``I'm frightfully anxious.''

``If you knew how great is a mother's love,'' Wendy told them
triumphantly, ``you would have no fear.'' She had now come to the part
that Peter hated.

``I do like a mother's love,'' said Tootles, hitting Nibs with a pillow.
``Do you like a mother's love, Nibs?''

``I do just,'' said Nibs, hitting back.

``You see,'' Wendy said complacently, ``our heroine knew that the mother
would always leave the window open for her children to fly back by; so
they stayed away for years and had a lovely time.''

``Did they ever go back?''

``Let us now,'' said Wendy, bracing herself up for her finest effort,
``take a peep into the future;'' and they all gave themselves the twist
that makes peeps into the future easier. ``Years have rolled by, and who
is this elegant lady of uncertain age alighting at London Station?''

``O Wendy, who is she?'' cried Nibs, every bit as excited as if he didn't
know.

``Can it be—yes—no—it is—the fair Wendy!''

``Oh!''

``And who are the two noble portly figures accompanying her, now grown
to man's estate? Can they be John and Michael? They are!''

``Oh!''

``‘See, dear brothers,' says Wendy pointing upwards, ‘there is the
window still standing open. Ah, now we are rewarded for our sublime
faith in a mother's love.' So up they flew to their mummy and daddy,
and pen cannot describe the happy scene, over which we draw a veil.''

That was the story, and they were as pleased with it as the fair
narrator herself. Everything just as it should be, you see. Off we skip
like the most heartless things in the world, which is what children
are, but so attractive; and we have an entirely selfish time, and then
when we have need of special attention we nobly return for it,
confident that we shall be rewarded instead of smacked.

So great indeed was their faith in a mother's love that they felt they
could afford to be callous for a bit longer.

But there was one there who knew better, and when Wendy finished he
uttered a hollow groan.

``What is it, Peter?'' she cried, running to him, thinking he was ill.
She felt him solicitously, lower down than his chest. ``Where is it,
Peter?''

``It isn't that kind of pain,'' Peter replied darkly.

``Then what kind is it?''

``Wendy, you are wrong about mothers.''

They all gathered round him in affright, so alarming was his agitation;
and with a fine candour he told them what he had hitherto concealed.

``Long ago,'' he said, ``I thought like you that my mother would always
keep the window open for me, so I stayed away for moons and moons and
moons, and then flew back; but the window was barred, for mother had
forgotten all about me, and there was another little boy sleeping in my
bed.''

I am not sure that this was true, but Peter thought it was true; and it
scared them.

``Are you sure mothers are like that?''

``Yes.''

So this was the truth about mothers. The toads!

Still it is best to be careful; and no one knows so quickly as a child
when he should give in. ``Wendy, let us go home,'' cried John and Michael
together.

``Yes,'' she said, clutching them.

``Not to-night?'' asked the lost boys bewildered. They knew in what they
called their hearts that one can get on quite well without a mother,
and that it is only the mothers who think you can't.

``At once,'' Wendy replied resolutely, for the horrible thought had come
to her: ``Perhaps mother is in half mourning by this time.''

This dread made her forgetful of what must be Peter's feelings, and she
said to him rather sharply, ``Peter, will you make the necessary
arrangements?''

``If you wish it,'' he replied, as coolly as if she had asked him to pass
the nuts.

Not so much as a sorry-to-lose-you between them! If she did not mind
the parting, he was going to show her, was Peter, that neither did he.

But of course he cared very much; and he was so full of wrath against
grown-ups, who, as usual, were spoiling everything, that as soon as he
got inside his tree he breathed intentionally quick short breaths at
the rate of about five to a second. He did this because there is a
saying in the Neverland that, every time you breathe, a grown-up dies;
and Peter was killing them off vindictively as fast as possible.

Then having given the necessary instructions to the redskins he
returned to the home, where an unworthy scene had been enacted in his
absence. Panic-stricken at the thought of losing Wendy the lost boys
had advanced upon her threateningly.

``It will be worse than before she came,'' they cried.

``We shan't let her go.''

``Let's keep her prisoner.''

``Ay, chain her up.''

In her extremity an instinct told her to which of them to turn.

``Tootles,'' she cried, ``I appeal to you.''

Was it not strange? She appealed to Tootles, quite the silliest one.

Grandly, however, did Tootles respond. For that one moment he dropped
his silliness and spoke with dignity.

``I am just Tootles,'' he said, ``and nobody minds me. But the first who
does not behave to Wendy like an English gentleman I will blood him
severely.''

He drew back his hanger; and for that instant his sun was at noon. The
others held back uneasily. Then Peter returned, and they saw at once
that they would get no support from him. He would keep no girl in the
Neverland against her will.

``Wendy,'' he said, striding up and down, ``I have asked the redskins to
guide you through the wood, as flying tires you so.''

``Thank you, Peter.''

``Then,'' he continued, in the short sharp voice of one accustomed to be
obeyed, ``Tinker Bell will take you across the sea. Wake her, Nibs.''

Nibs had to knock twice before he got an answer, though Tink had really
been sitting up in bed listening for some time.

``Who are you? How dare you? Go away,'' she cried.

``You are to get up, Tink,'' Nibs called, ``and take Wendy on a journey.''

Of course Tink had been delighted to hear that Wendy was going; but she
was jolly well determined not to be her courier, and she said so in
still more offensive language. Then she pretended to be asleep again.

``She says she won't!'' Nibs exclaimed, aghast at such insubordination,
whereupon Peter went sternly toward the young lady's chamber.

``Tink,'' he rapped out, ``if you don't get up and dress at once I will
open the curtains, and then we shall all see you in your _negligée_.''

This made her leap to the floor. ``Who said I wasn't getting up?'' she
cried.

In the meantime the boys were gazing very forlornly at Wendy, now
equipped with John and Michael for the journey. By this time they were
dejected, not merely because they were about to lose her, but also
because they felt that she was going off to something nice to which
they had not been invited. Novelty was beckoning to them as usual.

Crediting them with a nobler feeling Wendy melted.

``Dear ones,'' she said, ``if you will all come with me I feel almost sure
I can get my father and mother to adopt you.''

The invitation was meant specially for Peter, but each of the boys was
thinking exclusively of himself, and at once they jumped with joy.

``But won't they think us rather a handful?'' Nibs asked in the middle of
his jump.

``Oh no,'' said Wendy, rapidly thinking it out, ``it will only mean having
a few beds in the drawing-room; they can be hidden behind the screens
on first Thursdays.''

``Peter, can we go?'' they all cried imploringly. They took it for
granted that if they went he would go also, but really they scarcely
cared. Thus children are ever ready, when novelty knocks, to desert
their dearest ones.

``All right,'' Peter replied with a bitter smile, and immediately they
rushed to get their things.

``And now, Peter,'' Wendy said, thinking she had put everything right, ``I
am going to give you your medicine before you go.'' She loved to give
them medicine, and undoubtedly gave them too much. Of course it was
only water, but it was out of a bottle, and she always shook the bottle
and counted the drops, which gave it a certain medicinal quality. On
this occasion, however, she did not give Peter his draught, for just as
she had prepared it, she saw a look on his face that made her heart
sink.

``Get your things, Peter,'' she cried, shaking.

``No,'' he answered, pretending indifference, ``I am not going with you,
Wendy.''

``Yes, Peter.''

``No.''

To show that her departure would leave him unmoved, he skipped up and
down the room, playing gaily on his heartless pipes. She had to run
about after him, though it was rather undignified.

``To find your mother,'' she coaxed.

Now, if Peter had ever quite had a mother, he no longer missed her. He
could do very well without one. He had thought them out, and remembered
only their bad points.

``No, no,'' he told Wendy decisively; ``perhaps she would say I was old,
and I just want always to be a little boy and to have fun.''

``But, Peter—''

``No.''

And so the others had to be told.

``Peter isn't coming.''

Peter not coming! They gazed blankly at him, their sticks over their
backs, and on each stick a bundle. Their first thought was that if
Peter was not going he had probably changed his mind about letting them
go.

But he was far too proud for that. ``If you find your mothers,'' he said
darkly, ``I hope you will like them.''

The awful cynicism of this made an uncomfortable impression, and most
of them began to look rather doubtful. After all, their faces said,
were they not noodles to want to go?

``Now then,'' cried Peter, ``no fuss, no blubbering; good-bye, Wendy;'' and
he held out his hand cheerily, quite as if they must really go now, for
he had something important to do.

She had to take his hand, and there was no indication that he would
prefer a thimble.

``You will remember about changing your flannels, Peter?'' she said,
lingering over him. She was always so particular about their flannels.

``Yes.''

``And you will take your medicine?''

``Yes.''

That seemed to be everything, and an awkward pause followed. Peter,
however, was not the kind that breaks down before other people. ``Are
you ready, Tinker Bell?'' he called out.

``Ay, ay.''

``Then lead the way.''

Tink darted up the nearest tree; but no one followed her, for it was at
this moment that the pirates made their dreadful attack upon the
redskins. Above, where all had been so still, the air was rent with
shrieks and the clash of steel. Below, there was dead silence. Mouths
opened and remained open. Wendy fell on her knees, but her arms were
extended toward Peter. All arms were extended to him, as if suddenly
blown in his direction; they were beseeching him mutely not to desert
them. As for Peter, he seized his sword, the same he thought he had
slain Barbecue with, and the lust of battle was in his eye.




Chapter XII.
THE CHILDREN ARE CARRIED OFF


The pirate attack had been a complete surprise: a sure proof that the
unscrupulous Hook had conducted it improperly, for to surprise redskins
fairly is beyond the wit of the white man.

By all the unwritten laws of savage warfare it is always the redskin
who attacks, and with the wiliness of his race he does it just before
the dawn, at which time he knows the courage of the whites to be at its
lowest ebb. The white men have in the meantime made a rude stockade on
the summit of yonder undulating ground, at the foot of which a stream
runs, for it is destruction to be too far from water. There they await
the onslaught, the inexperienced ones clutching their revolvers and
treading on twigs, but the old hands sleeping tranquilly until just
before the dawn. Through the long black night the savage scouts
wriggle, snake-like, among the grass without stirring a blade. The
brushwood closes behind them, as silently as sand into which a mole has
dived. Not a sound is to be heard, save when they give vent to a
wonderful imitation of the lonely call of the coyote. The cry is
answered by other braves; and some of them do it even better than the
coyotes, who are not very good at it. So the chill hours wear on, and
the long suspense is horribly trying to the paleface who has to live
through it for the first time; but to the trained hand those ghastly
calls and still ghastlier silences are but an intimation of how the
night is marching.

That this was the usual procedure was so well known to Hook that in
disregarding it he cannot be excused on the plea of ignorance.

The Piccaninnies, on their part, trusted implicitly to his honour, and
their whole action of the night stands out in marked contrast to his.
They left nothing undone that was consistent with the reputation of
their tribe. With that alertness of the senses which is at once the
marvel and despair of civilised peoples, they knew that the pirates
were on the island from the moment one of them trod on a dry stick; and
in an incredibly short space of time the coyote cries began. Every foot
of ground between the spot where Hook had landed his forces and the
home under the trees was stealthily examined by braves wearing their
mocassins with the heels in front. They found only one hillock with a
stream at its base, so that Hook had no choice; here he must establish
himself and wait for just before the dawn. Everything being thus mapped
out with almost diabolical cunning, the main body of the redskins
folded their blankets around them, and in the phlegmatic manner that is
to them, the pearl of manhood squatted above the children's home,
awaiting the cold moment when they should deal pale death.

Here dreaming, though wide-awake, of the exquisite tortures to which
they were to put him at break of day, those confiding savages were
found by the treacherous Hook. From the accounts afterwards supplied by
such of the scouts as escaped the carnage, he does not seem even to
have paused at the rising ground, though it is certain that in that
grey light he must have seen it: no thought of waiting to be attacked
appears from first to last to have visited his subtle mind; he would
not even hold off till the night was nearly spent; on he pounded with
no policy but to fall to. What could the bewildered scouts do, masters
as they were of every war-like artifice save this one, but trot
helplessly after him, exposing themselves fatally to view, while they
gave pathetic utterance to the coyote cry.

Around the brave Tiger Lily were a dozen of her stoutest warriors, and
they suddenly saw the perfidious pirates bearing down upon them. Fell
from their eyes then the film through which they had looked at victory.
No more would they torture at the stake. For them the happy
hunting-grounds was now. They knew it; but as their father's sons they
acquitted themselves. Even then they had time to gather in a phalanx
that would have been hard to break had they risen quickly, but this
they were forbidden to do by the traditions of their race. It is
written that the noble savage must never express surprise in the
presence of the white. Thus terrible as the sudden appearance of the
pirates must have been to them, they remained stationary for a moment,
not a muscle moving; as if the foe had come by invitation. Then,
indeed, the tradition gallantly upheld, they seized their weapons, and
the air was torn with the war-cry; but it was now too late.

It is no part of ours to describe what was a massacre rather than a
fight. Thus perished many of the flower of the Piccaninny tribe. Not
all unavenged did they die, for with Lean Wolf fell Alf Mason, to
disturb the Spanish Main no more, and among others who bit the dust
were Geo. Scourie, Chas. Turley, and the Alsatian Foggerty. Turley fell
to the tomahawk of the terrible Panther, who ultimately cut a way
through the pirates with Tiger Lily and a small remnant of the tribe.

To what extent Hook is to blame for his tactics on this occasion is for
the historian to decide. Had he waited on the rising ground till the
proper hour he and his men would probably have been butchered; and in
judging him it is only fair to take this into account. What he should
perhaps have done was to acquaint his opponents that he proposed to
follow a new method. On the other hand, this, as destroying the element
of surprise, would have made his strategy of no avail, so that the
whole question is beset with difficulties. One cannot at least withhold
a reluctant admiration for the wit that had conceived so bold a scheme,
and the fell genius with which it was carried out.

What were his own feelings about himself at that triumphant moment?
Fain would his dogs have known, as breathing heavily and wiping their
cutlasses, they gathered at a discreet distance from his hook, and
squinted through their ferret eyes at this extraordinary man. Elation
must have been in his heart, but his face did not reflect it: ever a
dark and solitary enigma, he stood aloof from his followers in spirit
as in substance.

The night's work was not yet over, for it was not the redskins he had
come out to destroy; they were but the bees to be smoked, so that he
should get at the honey. It was Pan he wanted, Pan and Wendy and their
band, but chiefly Pan.

Peter was such a small boy that one tends to wonder at the man's hatred
of him. True he had flung Hook's arm to the crocodile, but even this
and the increased insecurity of life to which it led, owing to the
crocodile's pertinacity, hardly account for a vindictiveness so
relentless and malignant. The truth is that there was a something about
Peter which goaded the pirate captain to frenzy. It was not his
courage, it was not his engaging appearance, it was not—. There is no
beating about the bush, for we know quite well what it was, and have
got to tell. It was Peter's cockiness.

This had got on Hook's nerves; it made his iron claw twitch, and at
night it disturbed him like an insect. While Peter lived, the tortured
man felt that he was a lion in a cage into which a sparrow had come.

The question now was how to get down the trees, or how to get his dogs
down? He ran his greedy eyes over them, searching for the thinnest
ones. They wriggled uncomfortably, for they knew he would not scruple
to ram them down with poles.

In the meantime, what of the boys? We have seen them at the first clang
of the weapons, turned as it were into stone figures, open-mouthed, all
appealing with outstretched arms to Peter; and we return to them as
their mouths close, and their arms fall to their sides. The pandemonium
above has ceased almost as suddenly as it arose, passed like a fierce
gust of wind; but they know that in the passing it has determined their
fate.

Which side had won?

The pirates, listening avidly at the mouths of the trees, heard the
question put by every boy, and alas, they also heard Peter's answer.

``If the redskins have won,'' he said, ``they will beat the tom-tom; it is
always their sign of victory.''

Now Smee had found the tom-tom, and was at that moment sitting on it.
``You will never hear the tom-tom again,'' he muttered, but inaudibly of
course, for strict silence had been enjoined. To his amazement Hook
signed him to beat the tom-tom, and slowly there came to Smee an
understanding of the dreadful wickedness of the order. Never, probably,
had this simple man admired Hook so much.

Twice Smee beat upon the instrument, and then stopped to listen
gleefully.

``The tom-tom,'' the miscreants heard Peter cry; ``an Indian victory!''

The doomed children answered with a cheer that was music to the black
hearts above, and almost immediately they repeated their good-byes to
Peter. This puzzled the pirates, but all their other feelings were
swallowed by a base delight that the enemy were about to come up the
trees. They smirked at each other and rubbed their hands. Rapidly and
silently Hook gave his orders: one man to each tree, and the others to
arrange themselves in a line two yards apart.




Chapter XIII.
DO YOU BELIEVE IN FAIRIES?


The more quickly this horror is disposed of the better. The first to
emerge from his tree was Curly. He rose out of it into the arms of
Cecco, who flung him to Smee, who flung him to Starkey, who flung him
to Bill Jukes, who flung him to Noodler, and so he was tossed from one
to another till he fell at the feet of the black pirate. All the boys
were plucked from their trees in this ruthless manner; and several of
them were in the air at a time, like bales of goods flung from hand to
hand.

A different treatment was accorded to Wendy, who came last. With
ironical politeness Hook raised his hat to her, and, offering her his
arm, escorted her to the spot where the others were being gagged. He
did it with such an air, he was so frightfully _distingué_, that she
was too fascinated to cry out. She was only a little girl.

Perhaps it is tell-tale to divulge that for a moment Hook entranced
her, and we tell on her only because her slip led to strange results.
Had she haughtily unhanded him (and we should have loved to write it of
her), she would have been hurled through the air like the others, and
then Hook would probably not have been present at the tying of the
children; and had he not been at the tying he would not have discovered
Slightly's secret, and without the secret he could not presently have
made his foul attempt on Peter's life.

They were tied to prevent their flying away, doubled up with their
knees close to their ears; and for the trussing of them the black
pirate had cut a rope into nine equal pieces. All went well until
Slightly's turn came, when he was found to be like those irritating
parcels that use up all the string in going round and leave no tags
with which to tie a knot. The pirates kicked him in their rage, just as
you kick the parcel (though in fairness you should kick the string);
and strange to say it was Hook who told them to belay their violence.
His lip was curled with malicious triumph. While his dogs were merely
sweating because every time they tried to pack the unhappy lad tight in
one part he bulged out in another, Hook's master mind had gone far
beneath Slightly's surface, probing not for effects but for causes; and
his exultation showed that he had found them. Slightly, white to the
gills, knew that Hook had surprised his secret, which was this, that no
boy so blown out could use a tree wherein an average man need stick.
Poor Slightly, most wretched of all the children now, for he was in a
panic about Peter, bitterly regretted what he had done. Madly addicted
to the drinking of water when he was hot, he had swelled in consequence
to his present girth, and instead of reducing himself to fit his tree
he had, unknown to the others, whittled his tree to make it fit him.

Sufficient of this Hook guessed to persuade him that Peter at last lay
at his mercy, but no word of the dark design that now formed in the
subterranean caverns of his mind crossed his lips; he merely signed
that the captives were to be conveyed to the ship, and that he would be
alone.

How to convey them? Hunched up in their ropes they might indeed be
rolled down hill like barrels, but most of the way lay through a
morass. Again Hook's genius surmounted difficulties. He indicated that
the little house must be used as a conveyance. The children were flung
into it, four stout pirates raised it on their shoulders, the others
fell in behind, and singing the hateful pirate chorus the strange
procession set off through the wood. I don't know whether any of the
children were crying; if so, the singing drowned the sound; but as the
little house disappeared in the forest, a brave though tiny jet of
smoke issued from its chimney as if defying Hook.

Hook saw it, and it did Peter a bad service. It dried up any trickle of
pity for him that may have remained in the pirate's infuriated breast.

The first thing he did on finding himself alone in the fast falling
night was to tiptoe to Slightly's tree, and make sure that it provided
him with a passage. Then for long he remained brooding; his hat of ill
omen on the sward, so that any gentle breeze which had arisen might
play refreshingly through his hair. Dark as were his thoughts his blue
eyes were as soft as the periwinkle. Intently he listened for any sound
from the nether world, but all was as silent below as above; the house
under the ground seemed to be but one more empty tenement in the void.
Was that boy asleep, or did he stand waiting at the foot of Slightly's
tree, with his dagger in his hand?

There was no way of knowing, save by going down. Hook let his cloak
slip softly to the ground, and then biting his lips till a lewd blood
stood on them, he stepped into the tree. He was a brave man, but for a
moment he had to stop there and wipe his brow, which was dripping like
a candle. Then, silently, he let himself go into the unknown.

He arrived unmolested at the foot of the shaft, and stood still again,
biting at his breath, which had almost left him. As his eyes became
accustomed to the dim light various objects in the home under the trees
took shape; but the only one on which his greedy gaze rested, long
sought for and found at last, was the great bed. On the bed lay Peter
fast asleep.

Unaware of the tragedy being enacted above, Peter had continued, for a
little time after the children left, to play gaily on his pipes: no
doubt rather a forlorn attempt to prove to himself that he did not
care. Then he decided not to take his medicine, so as to grieve Wendy.
Then he lay down on the bed outside the coverlet, to vex her still
more; for she had always tucked them inside it, because you never know
that you may not grow chilly at the turn of the night. Then he nearly
cried; but it struck him how indignant she would be if he laughed
instead; so he laughed a haughty laugh and fell asleep in the middle of
it.

Sometimes, though not often, he had dreams, and they were more painful
than the dreams of other boys. For hours he could not be separated from
these dreams, though he wailed piteously in them. They had to do, I
think, with the riddle of his existence. At such times it had been
Wendy's custom to take him out of bed and sit with him on her lap,
soothing him in dear ways of her own invention, and when he grew calmer
to put him back to bed before he quite woke up, so that he should not
know of the indignity to which she had subjected him. But on this
occasion he had fallen at once into a dreamless sleep. One arm dropped
over the edge of the bed, one leg was arched, and the unfinished part
of his laugh was stranded on his mouth, which was open, showing the
little pearls.

Thus defenceless Hook found him. He stood silent at the foot of the
tree looking across the chamber at his enemy. Did no feeling of
compassion disturb his sombre breast? The man was not wholly evil; he
loved flowers (I have been told) and sweet music (he was himself no
mean performer on the harpsichord); and, let it be frankly admitted,
the idyllic nature of the scene stirred him profoundly. Mastered by his
better self he would have returned reluctantly up the tree, but for one
thing.

What stayed him was Peter's impertinent appearance as he slept. The
open mouth, the drooping arm, the arched knee: they were such a
personification of cockiness as, taken together, will never again, one
may hope, be presented to eyes so sensitive to their offensiveness.
They steeled Hook's heart. If his rage had broken him into a hundred
pieces every one of them would have disregarded the incident, and leapt
at the sleeper.

Though a light from the one lamp shone dimly on the bed, Hook stood in
darkness himself, and at the first stealthy step forward he discovered
an obstacle, the door of Slightly's tree. It did not entirely fill the
aperture, and he had been looking over it. Feeling for the catch, he
found to his fury that it was low down, beyond his reach. To his
disordered brain it seemed then that the irritating quality in Peter's
face and figure visibly increased, and he rattled the door and flung
himself against it. Was his enemy to escape him after all?

But what was that? The red in his eye had caught sight of Peter's
medicine standing on a ledge within easy reach. He fathomed what it was
straightaway, and immediately knew that the sleeper was in his power.

Lest he should be taken alive, Hook always carried about his person a
dreadful drug, blended by himself of all the death-dealing rings that
had come into his possession. These he had boiled down into a yellow
liquid quite unknown to science, which was probably the most virulent
poison in existence.

Five drops of this he now added to Peter's cup. His hand shook, but it
was in exultation rather than in shame. As he did it he avoided
glancing at the sleeper, but not lest pity should unnerve him; merely
to avoid spilling. Then one long gloating look he cast upon his victim,
and turning, wormed his way with difficulty up the tree. As he emerged
at the top he looked the very spirit of evil breaking from its hole.
Donning his hat at its most rakish angle, he wound his cloak around
him, holding one end in front as if to conceal his person from the
night, of which it was the blackest part, and muttering strangely to
himself, stole away through the trees.

Peter slept on. The light guttered and went out, leaving the tenement
in darkness; but still he slept. It must have been not less than ten
o'clock by the crocodile, when he suddenly sat up in his bed, wakened
by he knew not what. It was a soft cautious tapping on the door of his
tree.

Soft and cautious, but in that stillness it was sinister. Peter felt
for his dagger till his hand gripped it. Then he spoke.

``Who is that?''

For long there was no answer: then again the knock.

``Who are you?''

No answer.

He was thrilled, and he loved being thrilled. In two strides he reached
the door. Unlike Slightly's door, it filled the aperture, so that he
could not see beyond it, nor could the one knocking see him.

``I won't open unless you speak,'' Peter cried.

Then at last the visitor spoke, in a lovely bell-like voice.

``Let me in, Peter.''

It was Tink, and quickly he unbarred to her. She flew in excitedly, her
face flushed and her dress stained with mud.

``What is it?''

``Oh, you could never guess!'' she cried, and offered him three guesses.
``Out with it!'' he shouted, and in one ungrammatical sentence, as long
as the ribbons that conjurers pull from their mouths, she told of the
capture of Wendy and the boys.

Peter's heart bobbed up and down as he listened. Wendy bound, and on
the pirate ship; she who loved everything to be just so!

``I'll rescue her!'' he cried, leaping at his weapons. As he leapt he
thought of something he could do to please her. He could take his
medicine.

His hand closed on the fatal draught.

``No!'' shrieked Tinker Bell, who had heard Hook mutter about his deed as
he sped through the forest.

``Why not?''

``It is poisoned.''

``Poisoned? Who could have poisoned it?''

``Hook.''

``Don't be silly. How could Hook have got down here?''

Alas, Tinker Bell could not explain this, for even she did not know the
dark secret of Slightly's tree. Nevertheless Hook's words had left no
room for doubt. The cup was poisoned.

``Besides,'' said Peter, quite believing himself, ``I never fell asleep.''

He raised the cup. No time for words now; time for deeds; and with one
of her lightning movements Tink got between his lips and the draught,
and drained it to the dregs.

``Why, Tink, how dare you drink my medicine?''

But she did not answer. Already she was reeling in the air.

``What is the matter with you?'' cried Peter, suddenly afraid.

``It was poisoned, Peter,'' she told him softly; ``and now I am going to
be dead.''

``O Tink, did you drink it to save me?''

``Yes.''

``But why, Tink?''

Her wings would scarcely carry her now, but in reply she alighted on
his shoulder and gave his nose a loving bite. She whispered in his ear
``You silly ass,'' and then, tottering to her chamber, lay down on the
bed.

His head almost filled the fourth wall of her little room as he knelt
near her in distress. Every moment her light was growing fainter; and
he knew that if it went out she would be no more. She liked his tears
so much that she put out her beautiful finger and let them run over it.

Her voice was so low that at first he could not make out what she said.
Then he made it out. She was saying that she thought she could get well
again if children believed in fairies.

Peter flung out his arms. There were no children there, and it was
night time; but he addressed all who might be dreaming of the
Neverland, and who were therefore nearer to him than you think: boys
and girls in their nighties, and naked papooses in their baskets hung
from trees.

``Do you believe?'' he cried.

Tink sat up in bed almost briskly to listen to her fate.

She fancied she heard answers in the affirmative, and then again she
wasn't sure.

``What do you think?'' she asked Peter.

``If you believe,'' he shouted to them, ``clap your hands; don't let Tink
die.''

Many clapped.

Some didn't.

A few beasts hissed.

The clapping stopped suddenly; as if countless mothers had rushed to
their nurseries to see what on earth was happening; but already Tink
was saved. First her voice grew strong, then she popped out of bed,
then she was flashing through the room more merry and impudent than
ever. She never thought of thanking those who believed, but she would
have liked to get at the ones who had hissed.

``And now to rescue Wendy!''

The moon was riding in a cloudy heaven when Peter rose from his tree,
begirt with weapons and wearing little else, to set out upon his
perilous quest. It was not such a night as he would have chosen. He had
hoped to fly, keeping not far from the ground so that nothing unwonted
should escape his eyes; but in that fitful light to have flown low
would have meant trailing his shadow through the trees, thus disturbing
birds and acquainting a watchful foe that he was astir.

He regretted now that he had given the birds of the island such strange
names that they are very wild and difficult of approach.

There was no other course but to press forward in redskin fashion, at
which happily he was an adept. But in what direction, for he could not
be sure that the children had been taken to the ship? A light fall of
snow had obliterated all footmarks; and a deathly silence pervaded the
island, as if for a space Nature stood still in horror of the recent
carnage. He had taught the children something of the forest lore that
he had himself learned from Tiger Lily and Tinker Bell, and knew that
in their dire hour they were not likely to forget it. Slightly, if he
had an opportunity, would blaze the trees, for instance, Curly would
drop seeds, and Wendy would leave her handkerchief at some important
place. The morning was needed to search for such guidance, and he could
not wait. The upper world had called him, but would give no help.

The crocodile passed him, but not another living thing, not a sound,
not a movement; and yet he knew well that sudden death might be at the
next tree, or stalking him from behind.

He swore this terrible oath: ``Hook or me this time.''

Now he crawled forward like a snake, and again erect, he darted across
a space on which the moonlight played, one finger on his lip and his
dagger at the ready. He was frightfully happy.




Chapter XIV.
THE PIRATE SHIP


One green light squinting over Kidd's Creek, which is near the mouth of
the pirate river, marked where the brig, the _Jolly Roger_, lay, low in
the water; a rakish-looking craft foul to the hull, every beam in her
detestable, like ground strewn with mangled feathers. She was the
cannibal of the seas, and scarce needed that watchful eye, for she
floated immune in the horror of her name.

She was wrapped in the blanket of night, through which no sound from
her could have reached the shore. There was little sound, and none
agreeable save the whir of the ship's sewing machine at which Smee sat,
ever industrious and obliging, the essence of the commonplace, pathetic
Smee. I know not why he was so infinitely pathetic, unless it were
because he was so pathetically unaware of it; but even strong men had
to turn hastily from looking at him, and more than once on summer
evenings he had touched the fount of Hook's tears and made it flow. Of
this, as of almost everything else, Smee was quite unconscious.

A few of the pirates leant over the bulwarks, drinking in the miasma of
the night; others sprawled by barrels over games of dice and cards; and
the exhausted four who had carried the little house lay prone on the
deck, where even in their sleep they rolled skillfully to this side or
that out of Hook's reach, lest he should claw them mechanically in
passing.

Hook trod the deck in thought. O man unfathomable. It was his hour of
triumph. Peter had been removed for ever from his path, and all the
other boys were in the brig, about to walk the plank. It was his
grimmest deed since the days when he had brought Barbecue to heel; and
knowing as we do how vain a tabernacle is man, could we be surprised
had he now paced the deck unsteadily, bellied out by the winds of his
success?

But there was no elation in his gait, which kept pace with the action
of his sombre mind. Hook was profoundly dejected.

He was often thus when communing with himself on board ship in the
quietude of the night. It was because he was so terribly alone. This
inscrutable man never felt more alone than when surrounded by his dogs.
They were socially inferior to him.

Hook was not his true name. To reveal who he really was would even at
this date set the country in a blaze; but as those who read between the
lines must already have guessed, he had been at a famous public school;
and its traditions still clung to him like garments, with which indeed
they are largely concerned. Thus it was offensive to him even now to
board a ship in the same dress in which he grappled her, and he still
adhered in his walk to the school's distinguished slouch. But above all
he retained the passion for good form.

Good form! However much he may have degenerated, he still knew that
this is all that really matters.

From far within him he heard a creaking as of rusty portals, and
through them came a stern tap-tap-tap, like hammering in the night when
one cannot sleep. ``Have you been good form to-day?'' was their eternal
question.

``Fame, fame, that glittering bauble, it is mine,'' he cried.

``Is it quite good form to be distinguished at anything?'' the tap-tap
from his school replied.

``I am the only man whom Barbecue feared,'' he urged, ``and Flint feared
Barbecue.''

``Barbecue, Flint—what house?'' came the cutting retort.

Most disquieting reflection of all, was it not bad form to think about
good form?

His vitals were tortured by this problem. It was a claw within him
sharper than the iron one; and as it tore him, the perspiration dripped
down his tallow countenance and streaked his doublet. Ofttimes he drew
his sleeve across his face, but there was no damming that trickle.

Ah, envy not Hook.

There came to him a presentiment of his early dissolution. It was as if
Peter's terrible oath had boarded the ship. Hook felt a gloomy desire
to make his dying speech, lest presently there should be no time for
it.

``Better for Hook,'' he cried, ``if he had had less ambition!'' It was in
his darkest hours only that he referred to himself in the third person.

``No little children to love me!''

Strange that he should think of this, which had never troubled him
before; perhaps the sewing machine brought it to his mind. For long he
muttered to himself, staring at Smee, who was hemming placidly, under
the conviction that all children feared him.

Feared him! Feared Smee! There was not a child on board the brig that
night who did not already love him. He had said horrid things to them
and hit them with the palm of his hand, because he could not hit with
his fist, but they had only clung to him the more. Michael had tried on
his spectacles.

To tell poor Smee that they thought him lovable! Hook itched to do it,
but it seemed too brutal. Instead, he revolved this mystery in his
mind: why do they find Smee lovable? He pursued the problem like the
sleuth-hound that he was. If Smee was lovable, what was it that made
him so? A terrible answer suddenly presented itself—``Good form?''

Had the bo'sun good form without knowing it, which is the best form of
all?

He remembered that you have to prove you don't know you have it before
you are eligible for Pop.

With a cry of rage he raised his iron hand over Smee's head; but he did
not tear. What arrested him was this reflection:

``To claw a man because he is good form, what would that be?''

``Bad form!''

The unhappy Hook was as impotent as he was damp, and he fell forward
like a cut flower.

His dogs thinking him out of the way for a time, discipline instantly
relaxed; and they broke into a bacchanalian dance, which brought him to
his feet at once, all traces of human weakness gone, as if a bucket of
water had passed over him.

``Quiet, you scugs,'' he cried, ``or I'll cast anchor in you;'' and at once
the din was hushed. ``Are all the children chained, so that they cannot
fly away?''

``Ay, ay.''

``Then hoist them up.''

The wretched prisoners were dragged from the hold, all except Wendy,
and ranged in line in front of him. For a time he seemed unconscious of
their presence. He lolled at his ease, humming, not unmelodiously,
snatches of a rude song, and fingering a pack of cards. Ever and anon
the light from his cigar gave a touch of colour to his face.

``Now then, bullies,'' he said briskly, ``six of you walk the plank
to-night, but I have room for two cabin boys. Which of you is it to
be?''

``Don't irritate him unnecessarily,'' had been Wendy's instructions in
the hold; so Tootles stepped forward politely. Tootles hated the idea
of signing under such a man, but an instinct told him that it would be
prudent to lay the responsibility on an absent person; and though a
somewhat silly boy, he knew that mothers alone are always willing to be
the buffer. All children know this about mothers, and despise them for
it, but make constant use of it.

So Tootles explained prudently, ``You see, sir, I don't think my mother
would like me to be a pirate. Would your mother like you to be a
pirate, Slightly?''

He winked at Slightly, who said mournfully, ``I don't think so,'' as if
he wished things had been otherwise. ``Would your mother like you to be
a pirate, Twin?''

``I don't think so,'' said the first twin, as clever as the others.
``Nibs, would—''

``Stow this gab,'' roared Hook, and the spokesmen were dragged back.
``You, boy,'' he said, addressing John, ``you look as if you had a little
pluck in you. Didst never want to be a pirate, my hearty?''

Now John had sometimes experienced this hankering at maths. prep.; and
he was struck by Hook's picking him out.

``I once thought of calling myself Red-handed Jack,'' he said
diffidently.

``And a good name too. We'll call you that here, bully, if you join.''

``What do you think, Michael?'' asked John.

``What would you call me if I join?'' Michael demanded.

``Blackbeard Joe.''

Michael was naturally impressed. ``What do you think, John?'' He wanted
John to decide, and John wanted him to decide.

``Shall we still be respectful subjects of the King?'' John inquired.

Through Hook's teeth came the answer: ``You would have to swear, ‘Down
with the King.'''

Perhaps John had not behaved very well so far, but he shone out now.

``Then I refuse,'' he cried, banging the barrel in front of Hook.

``And I refuse,'' cried Michael.

``Rule Britannia!'' squeaked Curly.

The infuriated pirates buffeted them in the mouth; and Hook roared out,
``That seals your doom. Bring up their mother. Get the plank ready.''

They were only boys, and they went white as they saw Jukes and Cecco
preparing the fatal plank. But they tried to look brave when Wendy was
brought up.

No words of mine can tell you how Wendy despised those pirates. To the
boys there was at least some glamour in the pirate calling; but all
that she saw was that the ship had not been tidied for years. There was
not a porthole on the grimy glass of which you might not have written
with your finger ``Dirty pig''; and she had already written it on
several. But as the boys gathered round her she had no thought, of
course, save for them.

``So, my beauty,'' said Hook, as if he spoke in syrup, ``you are to see
your children walk the plank.''

Fine gentlemen though he was, the intensity of his communings had
soiled his ruff, and suddenly he knew that she was gazing at it. With a
hasty gesture he tried to hide it, but he was too late.

``Are they to die?'' asked Wendy, with a look of such frightful contempt
that he nearly fainted.

``They are,'' he snarled. ``Silence all,'' he called gloatingly, ``for a
mother's last words to her children.''

At this moment Wendy was grand. ``These are my last words, dear boys,''
she said firmly. ``I feel that I have a message to you from your real
mothers, and it is this: ‘We hope our sons will die like English
gentlemen.'''

Even the pirates were awed, and Tootles cried out hysterically, ``I am
going to do what my mother hopes. What are you to do, Nibs?''

``What my mother hopes. What are you to do, Twin?''

``What my mother hopes. John, what are—''

But Hook had found his voice again.

``Tie her up!'' he shouted.

It was Smee who tied her to the mast. ``See here, honey,'' he whispered,
``I'll save you if you promise to be my mother.''

But not even for Smee would she make such a promise. ``I would almost
rather have no children at all,'' she said disdainfully.

It is sad to know that not a boy was looking at her as Smee tied her to
the mast; the eyes of all were on the plank: that last little walk they
were about to take. They were no longer able to hope that they would
walk it manfully, for the capacity to think had gone from them; they
could stare and shiver only.

Hook smiled on them with his teeth closed, and took a step toward
Wendy. His intention was to turn her face so that she should see the
boys walking the plank one by one. But he never reached her, he never
heard the cry of anguish he hoped to wring from her. He heard something
else instead.

It was the terrible tick-tick of the crocodile.

They all heard it—pirates, boys, Wendy; and immediately every head was
blown in one direction; not to the water whence the sound proceeded,
but toward Hook. All knew that what was about to happen concerned him
alone, and that from being actors they were suddenly become spectators.

Very frightful was it to see the change that came over him. It was as
if he had been clipped at every joint. He fell in a little heap.

The sound came steadily nearer; and in advance of it came this ghastly
thought, ``The crocodile is about to board the ship!''

Even the iron claw hung inactive; as if knowing that it was no
intrinsic part of what the attacking force wanted. Left so fearfully
alone, any other man would have lain with his eyes shut where he fell:
but the gigantic brain of Hook was still working, and under its
guidance he crawled on the knees along the deck as far from the sound
as he could go. The pirates respectfully cleared a passage for him, and
it was only when he brought up against the bulwarks that he spoke.

``Hide me!'' he cried hoarsely.

They gathered round him, all eyes averted from the thing that was
coming aboard. They had no thought of fighting it. It was Fate.

Only when Hook was hidden from them did curiosity loosen the limbs of
the boys so that they could rush to the ship's side to see the
crocodile climbing it. Then they got the strangest surprise of the
Night of Nights; for it was no crocodile that was coming to their aid.
It was Peter.

He signed to them not to give vent to any cry of admiration that might
rouse suspicion. Then he went on ticking.




Chapter XV.
``HOOK OR ME THIS TIME''


Odd things happen to all of us on our way through life without our
noticing for a time that they have happened. Thus, to take an instance,
we suddenly discover that we have been deaf in one ear for we don't
know how long, but, say, half an hour. Now such an experience had come
that night to Peter. When last we saw him he was stealing across the
island with one finger to his lips and his dagger at the ready. He had
seen the crocodile pass by without noticing anything peculiar about it,
but by and by he remembered that it had not been ticking. At first he
thought this eerie, but soon concluded rightly that the clock had run
down.

Without giving a thought to what might be the feelings of a
fellow-creature thus abruptly deprived of its closest companion, Peter
began to consider how he could turn the catastrophe to his own use; and
he decided to tick, so that wild beasts should believe he was the
crocodile and let him pass unmolested. He ticked superbly, but with one
unforeseen result. The crocodile was among those who heard the sound,
and it followed him, though whether with the purpose of regaining what
it had lost, or merely as a friend under the belief that it was again
ticking itself, will never be certainly known, for, like slaves to a
fixed idea, it was a stupid beast.

Peter reached the shore without mishap, and went straight on, his legs
encountering the water as if quite unaware that they had entered a new
element. Thus many animals pass from land to water, but no other human
of whom I know. As he swam he had but one thought: ``Hook or me this
time.'' He had ticked so long that he now went on ticking without
knowing that he was doing it. Had he known he would have stopped, for
to board the brig by help of the tick, though an ingenious idea, had
not occurred to him.

On the contrary, he thought he had scaled her side as noiseless as a
mouse; and he was amazed to see the pirates cowering from him, with
Hook in their midst as abject as if he had heard the crocodile.

The crocodile! No sooner did Peter remember it than he heard the
ticking. At first he thought the sound did come from the crocodile, and
he looked behind him swiftly. Then he realised that he was doing it
himself, and in a flash he understood the situation. ``How clever of
me!'' he thought at once, and signed to the boys not to burst into
applause.

It was at this moment that Ed Teynte the quartermaster emerged from the
forecastle and came along the deck. Now, reader, time what happened by
your watch. Peter struck true and deep. John clapped his hands on the
ill-fated pirate's mouth to stifle the dying groan. He fell forward.
Four boys caught him to prevent the thud. Peter gave the signal, and
the carrion was cast overboard. There was a splash, and then silence.
How long has it taken?

``One!'' (Slightly had begun to count.)

None too soon, Peter, every inch of him on tiptoe, vanished into the
cabin; for more than one pirate was screwing up his courage to look
round. They could hear each other's distressed breathing now, which
showed them that the more terrible sound had passed.

``It's gone, captain,'' Smee said, wiping off his spectacles. ``All's
still again.''

Slowly Hook let his head emerge from his ruff, and listened so intently
that he could have caught the echo of the tick. There was not a sound,
and he drew himself up firmly to his full height.

``Then here's to Johnny Plank!'' he cried brazenly, hating the boys more
than ever because they had seen him unbend. He broke into the
villainous ditty:

``Yo ho, yo ho, the frisky plank,
    You walks along it so,
Till it goes down and you goes down
    To Davy Jones below!''

To terrorise the prisoners the more, though with a certain loss of
dignity, he danced along an imaginary plank, grimacing at them as he
sang; and when he finished he cried, ``Do you want a touch of the cat
before you walk the plank?''

At that they fell on their knees. ``No, no!'' they cried so piteously
that every pirate smiled.

``Fetch the cat, Jukes,'' said Hook; ``it's in the cabin.''

The cabin! Peter was in the cabin! The children gazed at each other.

``Ay, ay,'' said Jukes blithely, and he strode into the cabin. They
followed him with their eyes; they scarce knew that Hook had resumed
his song, his dogs joining in with him:

``Yo ho, yo ho, the scratching cat,
    Its tails are nine, you know,
And when they're writ upon your back—''

What was the last line will never be known, for of a sudden the song
was stayed by a dreadful screech from the cabin. It wailed through the
ship, and died away. Then was heard a crowing sound which was well
understood by the boys, but to the pirates was almost more eerie than
the screech.

``What was that?'' cried Hook.

``Two,'' said Slightly solemnly.

The Italian Cecco hesitated for a moment and then swung into the cabin.
He tottered out, haggard.

``What's the matter with Bill Jukes, you dog?'' hissed Hook, towering
over him.

``The matter wi' him is he's dead, stabbed,'' replied Cecco in a hollow
voice.

``Bill Jukes dead!'' cried the startled pirates.

``The cabin's as black as a pit,'' Cecco said, almost gibbering, ``but
there is something terrible in there: the thing you heard crowing.''

The exultation of the boys, the lowering looks of the pirates, both
were seen by Hook.

``Cecco,'' he said in his most steely voice, ``go back and fetch me out
that doodle-doo.''

Cecco, bravest of the brave, cowered before his captain, crying ``No,
no''; but Hook was purring to his claw.

``Did you say you would go, Cecco?'' he said musingly.

Cecco went, first flinging his arms despairingly. There was no more
singing, all listened now; and again came a death-screech and again a
crow.

No one spoke except Slightly. ``Three,'' he said.

Hook rallied his dogs with a gesture. ``'S'death and odds fish,'' he
thundered, ``who is to bring me that doodle-doo?''

``Wait till Cecco comes out,'' growled Starkey, and the others took up
the cry.

``I think I heard you volunteer, Starkey,'' said Hook, purring again.

``No, by thunder!'' Starkey cried.

``My hook thinks you did,'' said Hook, crossing to him. ``I wonder if it
would not be advisable, Starkey, to humour the hook?''

``I'll swing before I go in there,'' replied Starkey doggedly, and again
he had the support of the crew.

``Is this mutiny?'' asked Hook more pleasantly than ever. ``Starkey's
ringleader!''

``Captain, mercy!'' Starkey whimpered, all of a tremble now.

``Shake hands, Starkey,'' said Hook, proffering his claw.

Starkey looked round for help, but all deserted him. As he backed up
Hook advanced, and now the red spark was in his eye. With a despairing
scream the pirate leapt upon Long Tom and precipitated himself into the
sea.

``Four,'' said Slightly.

``And now,'' Hook said courteously, ``did any other gentlemen say mutiny?''
Seizing a lantern and raising his claw with a menacing gesture, ``I'll
bring out that doodle-doo myself,'' he said, and sped into the cabin.

``Five.'' How Slightly longed to say it. He wetted his lips to be ready,
but Hook came staggering out, without his lantern.

``Something blew out the light,'' he said a little unsteadily.

``Something!'' echoed Mullins.

``What of Cecco?'' demanded Noodler.

``He's as dead as Jukes,'' said Hook shortly.

His reluctance to return to the cabin impressed them all unfavourably,
and the mutinous sounds again broke forth. All pirates are
superstitious, and Cookson cried, ``They do say the surest sign a ship's
accurst is when there's one on board more than can be accounted for.''

``I've heard,'' muttered Mullins, ``he always boards the pirate craft
last. Had he a tail, captain?''

``They say,'' said another, looking viciously at Hook, ``that when he
comes it's in the likeness of the wickedest man aboard.''

``Had he a hook, captain?'' asked Cookson insolently; and one after
another took up the cry, ``The ship's doomed!'' At this the children
could not resist raising a cheer. Hook had well-nigh forgotten his
prisoners, but as he swung round on them now his face lit up again.

``Lads,'' he cried to his crew, ``now here's a notion. Open the cabin door
and drive them in. Let them fight the doodle-doo for their lives. If
they kill him, we're so much the better; if he kills them, we're none
the worse.''

For the last time his dogs admired Hook, and devotedly they did his
bidding. The boys, pretending to struggle, were pushed into the cabin
and the door was closed on them.

``Now, listen!'' cried Hook, and all listened. But not one dared to face
the door. Yes, one, Wendy, who all this time had been bound to the
mast. It was for neither a scream nor a crow that she was watching, it
was for the reappearance of Peter.

She had not long to wait. In the cabin he had found the thing for which
he had gone in search: the key that would free the children of their
manacles, and now they all stole forth, armed with such weapons as they
could find. First signing them to hide, Peter cut Wendy's bonds, and
then nothing could have been easier than for them all to fly off
together; but one thing barred the way, an oath, ``Hook or me this
time.'' So when he had freed Wendy, he whispered for her to conceal
herself with the others, and himself took her place by the mast, her
cloak around him so that he should pass for her. Then he took a great
breath and crowed.

To the pirates it was a voice crying that all the boys lay slain in the
cabin; and they were panic-stricken. Hook tried to hearten them; but
like the dogs he had made them they showed him their fangs, and he knew
that if he took his eyes off them now they would leap at him.

``Lads,'' he said, ready to cajole or strike as need be, but never
quailing for an instant, ``I've thought it out. There's a Jonah aboard.''

``Ay,'' they snarled, ``a man wi' a hook.''

``No, lads, no, it's the girl. Never was luck on a pirate ship wi' a
woman on board. We'll right the ship when she's gone.''

Some of them remembered that this had been a saying of Flint's. ``It's
worth trying,'' they said doubtfully.

``Fling the girl overboard,'' cried Hook; and they made a rush at the
figure in the cloak.

``There's none can save you now, missy,'' Mullins hissed jeeringly.

``There's one,'' replied the figure.

``Who's that?''

``Peter Pan the avenger!'' came the terrible answer; and as he spoke
Peter flung off his cloak. Then they all knew who 'twas that had been
undoing them in the cabin, and twice Hook essayed to speak and twice he
failed. In that frightful moment I think his fierce heart broke.

At last he cried, ``Cleave him to the brisket!'' but without conviction.

``Down, boys, and at them!'' Peter's voice rang out; and in another
moment the clash of arms was resounding through the ship. Had the
pirates kept together it is certain that they would have won; but the
onset came when they were still unstrung, and they ran hither and
thither, striking wildly, each thinking himself the last survivor of
the crew. Man to man they were the stronger; but they fought on the
defensive only, which enabled the boys to hunt in pairs and choose
their quarry. Some of the miscreants leapt into the sea; others hid in
dark recesses, where they were found by Slightly, who did not fight,
but ran about with a lantern which he flashed in their faces, so that
they were half blinded and fell as an easy prey to the reeking swords
of the other boys. There was little sound to be heard but the clang of
weapons, an occasional screech or splash, and Slightly monotonously
counting—five—six—seven—eight—nine—ten—eleven.

I think all were gone when a group of savage boys surrounded Hook, who
seemed to have a charmed life, as he kept them at bay in that circle of
fire. They had done for his dogs, but this man alone seemed to be a
match for them all. Again and again they closed upon him, and again and
again he hewed a clear space. He had lifted up one boy with his hook,
and was using him as a buckler, when another, who had just passed his
sword through Mullins, sprang into the fray.

``Put up your swords, boys,'' cried the newcomer, ``this man is mine.''

Thus suddenly Hook found himself face to face with Peter. The others
drew back and formed a ring around them.

For long the two enemies looked at one another, Hook shuddering
slightly, and Peter with the strange smile upon his face.

``So, Pan,'' said Hook at last, ``this is all your doing.''

``Ay, James Hook,'' came the stern answer, ``it is all my doing.''

``Proud and insolent youth,'' said Hook, ``prepare to meet thy doom.''

``Dark and sinister man,'' Peter answered, ``have at thee.''

Without more words they fell to, and for a space there was no advantage
to either blade. Peter was a superb swordsman, and parried with
dazzling rapidity; ever and anon he followed up a feint with a lunge
that got past his foe's defence, but his shorter reach stood him in ill
stead, and he could not drive the steel home. Hook, scarcely his
inferior in brilliancy, but not quite so nimble in wrist play, forced
him back by the weight of his onset, hoping suddenly to end all with a
favourite thrust, taught him long ago by Barbecue at Rio; but to his
astonishment he found this thrust turned aside again and again. Then he
sought to close and give the quietus with his iron hook, which all this
time had been pawing the air; but Peter doubled under it and, lunging
fiercely, pierced him in the ribs. At the sight of his own blood, whose
peculiar colour, you remember, was offensive to him, the sword fell
from Hook's hand, and he was at Peter's mercy.

``Now!'' cried all the boys, but with a magnificent gesture Peter invited
his opponent to pick up his sword. Hook did so instantly, but with a
tragic feeling that Peter was showing good form.

Hitherto he had thought it was some fiend fighting him, but darker
suspicions assailed him now.

``Pan, who and what art thou?'' he cried huskily.

``I'm youth, I'm joy,'' Peter answered at a venture, ``I'm a little bird
that has broken out of the egg.''

This, of course, was nonsense; but it was proof to the unhappy Hook
that Peter did not know in the least who or what he was, which is the
very pinnacle of good form.

``To't again,'' he cried despairingly.

He fought now like a human flail, and every sweep of that terrible
sword would have severed in twain any man or boy who obstructed it; but
Peter fluttered round him as if the very wind it made blew him out of
the danger zone. And again and again he darted in and pricked.

Hook was fighting now without hope. That passionate breast no longer
asked for life; but for one boon it craved: to see Peter show bad form
before it was cold forever.

Abandoning the fight he rushed into the powder magazine and fired it.

``In two minutes,'' he cried, ``the ship will be blown to pieces.''

Now, now, he thought, true form will show.

But Peter issued from the powder magazine with the shell in his hands,
and calmly flung it overboard.

What sort of form was Hook himself showing? Misguided man though he
was, we may be glad, without sympathising with him, that in the end he
was true to the traditions of his race. The other boys were flying
around him now, flouting, scornful; and he staggered about the deck
striking up at them impotently, his mind was no longer with them; it
was slouching in the playing fields of long ago, or being sent up for
good, or watching the wall-game from a famous wall. And his shoes were
right, and his waistcoat was right, and his tie was right, and his
socks were right.

James Hook, thou not wholly unheroic figure, farewell.

For we have come to his last moment.

Seeing Peter slowly advancing upon him through the air with dagger
poised, he sprang upon the bulwarks to cast himself into the sea. He
did not know that the crocodile was waiting for him; for we purposely
stopped the clock that this knowledge might be spared him: a little
mark of respect from us at the end.

He had one last triumph, which I think we need not grudge him. As he
stood on the bulwark looking over his shoulder at Peter gliding through
the air, he invited him with a gesture to use his foot. It made Peter
kick instead of stab.

At last Hook had got the boon for which he craved.

``Bad form,'' he cried jeeringly, and went content to the crocodile.

Thus perished James Hook.

``Seventeen,'' Slightly sang out; but he was not quite correct in his
figures. Fifteen paid the penalty for their crimes that night; but two
reached the shore: Starkey to be captured by the redskins, who made him
nurse for all their papooses, a melancholy come-down for a pirate; and
Smee, who henceforth wandered about the world in his spectacles, making
a precarious living by saying he was the only man that Jas. Hook had
feared.

Wendy, of course, had stood by taking no part in the fight, though
watching Peter with glistening eyes; but now that all was over she
became prominent again. She praised them equally, and shuddered
delightfully when Michael showed her the place where he had killed one;
and then she took them into Hook's cabin and pointed to his watch which
was hanging on a nail. It said ``half-past one!''

The lateness of the hour was almost the biggest thing of all. She got
them to bed in the pirates' bunks pretty quickly, you may be sure; all
but Peter, who strutted up and down on the deck, until at last he fell
asleep by the side of Long Tom. He had one of his dreams that night,
and cried in his sleep for a long time, and Wendy held him tightly.




Chapter XVI.
THE RETURN HOME


By three bells that morning they were all stirring their stumps; for
there was a big sea running; and Tootles, the bo'sun, was among them,
with a rope's end in his hand and chewing tobacco. They all donned
pirate clothes cut off at the knee, shaved smartly, and tumbled up,
with the true nautical roll and hitching their trousers.

It need not be said who was the captain. Nibs and John were first and
second mate. There was a woman aboard. The rest were tars before the
mast, and lived in the fo'c'sle. Peter had already lashed himself to
the wheel; but he piped all hands and delivered a short address to
them; said he hoped they would do their duty like gallant hearties, but
that he knew they were the scum of Rio and the Gold Coast, and if they
snapped at him he would tear them. The bluff strident words struck the
note sailors understood, and they cheered him lustily. Then a few sharp
orders were given, and they turned the ship round, and nosed her for
the mainland.

Captain Pan calculated, after consulting the ship's chart, that if this
weather lasted they should strike the Azores about the 21st of June,
after which it would save time to fly.

Some of them wanted it to be an honest ship and others were in favour
of keeping it a pirate; but the captain treated them as dogs, and they
dared not express their wishes to him even in a round robin. Instant
obedience was the only safe thing. Slightly got a dozen for looking
perplexed when told to take soundings. The general feeling was that
Peter was honest just now to lull Wendy's suspicions, but that there
might be a change when the new suit was ready, which, against her will,
she was making for him out of some of Hook's wickedest garments. It was
afterwards whispered among them that on the first night he wore this
suit he sat long in the cabin with Hook's cigar-holder in his mouth and
one hand clenched, all but for the forefinger, which he bent and held
threateningly aloft like a hook.

Instead of watching the ship, however, we must now return to that
desolate home from which three of our characters had taken heartless
flight so long ago. It seems a shame to have neglected No. 14 all this
time; and yet we may be sure that Mrs. Darling does not blame us. If we
had returned sooner to look with sorrowful sympathy at her, she would
probably have cried, ``Don't be silly; what do I matter? Do go back and
keep an eye on the children.'' So long as mothers are like this their
children will take advantage of them; and they may lay to that.

Even now we venture into that familiar nursery only because its lawful
occupants are on their way home; we are merely hurrying on in advance
of them to see that their beds are properly aired and that Mr. and Mrs.
Darling do not go out for the evening. We are no more than servants.
Why on earth should their beds be properly aired, seeing that they left
them in such a thankless hurry? Would it not serve them jolly well
right if they came back and found that their parents were spending the
week-end in the country? It would be the moral lesson they have been in
need of ever since we met them; but if we contrived things in this way
Mrs. Darling would never forgive us.

One thing I should like to do immensely, and that is to tell her, in
the way authors have, that the children are coming back, that indeed
they will be here on Thursday week. This would spoil so completely the
surprise to which Wendy and John and Michael are looking forward. They
have been planning it out on the ship: mother's rapture, father's shout
of joy, Nana's leap through the air to embrace them first, when what
they ought to be prepared for is a good hiding. How delicious to spoil
it all by breaking the news in advance; so that when they enter grandly
Mrs. Darling may not even offer Wendy her mouth, and Mr. Darling may
exclaim pettishly, ``Dash it all, here are those boys again.'' However,
we should get no thanks even for this. We are beginning to know Mrs.
Darling by this time, and may be sure that she would upbraid us for
depriving the children of their little pleasure.

``But, my dear madam, it is ten days till Thursday week; so that by
telling you what's what, we can save you ten days of unhappiness.''

``Yes, but at what a cost! By depriving the children of ten minutes of
delight.''

``Oh, if you look at it in that way!''

``What other way is there in which to look at it?''

You see, the woman had no proper spirit. I had meant to say
extraordinarily nice things about her; but I despise her, and not one
of them will I say now. She does not really need to be told to have
things ready, for they are ready. All the beds are aired, and she never
leaves the house, and observe, the window is open. For all the use we
are to her, we might well go back to the ship. However, as we are here
we may as well stay and look on. That is all we are, lookers-on. Nobody
really wants us. So let us watch and say jaggy things, in the hope that
some of them will hurt.

The only change to be seen in the night-nursery is that between nine
and six the kennel is no longer there. When the children flew away, Mr.
Darling felt in his bones that all the blame was his for having chained
Nana up, and that from first to last she had been wiser than he. Of
course, as we have seen, he was quite a simple man; indeed he might
have passed for a boy again if he had been able to take his baldness
off; but he had also a noble sense of justice and a lion's courage to
do what seemed right to him; and having thought the matter out with
anxious care after the flight of the children, he went down on all
fours and crawled into the kennel. To all Mrs. Darling's dear
invitations to him to come out he replied sadly but firmly:

``No, my own one, this is the place for me.''

In the bitterness of his remorse he swore that he would never leave the
kennel until his children came back. Of course this was a pity; but
whatever Mr. Darling did he had to do in excess, otherwise he soon gave
up doing it. And there never was a more humble man than the once proud
George Darling, as he sat in the kennel of an evening talking with his
wife of their children and all their pretty ways.

Very touching was his deference to Nana. He would not let her come into
the kennel, but on all other matters he followed her wishes implicitly.

Every morning the kennel was carried with Mr. Darling in it to a cab,
which conveyed him to his office, and he returned home in the same way
at six. Something of the strength of character of the man will be seen
if we remember how sensitive he was to the opinion of neighbours: this
man whose every movement now attracted surprised attention. Inwardly he
must have suffered torture; but he preserved a calm exterior even when
the young criticised his little home, and he always lifted his hat
courteously to any lady who looked inside.

It may have been Quixotic, but it was magnificent. Soon the inward
meaning of it leaked out, and the great heart of the public was
touched. Crowds followed the cab, cheering it lustily; charming girls
scaled it to get his autograph; interviews appeared in the better class
of papers, and society invited him to dinner and added, ``Do come in the
kennel.''

On that eventful Thursday week, Mrs. Darling was in the night-nursery
awaiting George's return home; a very sad-eyed woman. Now that we look
at her closely and remember the gaiety of her in the old days, all gone
now just because she has lost her babes, I find I won't be able to say
nasty things about her after all. If she was too fond of her rubbishy
children, she couldn't help it. Look at her in her chair, where she has
fallen asleep. The corner of her mouth, where one looks first, is
almost withered up. Her hand moves restlessly on her breast as if she
had a pain there. Some like Peter best, and some like Wendy best, but I
like her best. Suppose, to make her happy, we whisper to her in her
sleep that the brats are coming back. They are really within two miles
of the window now, and flying strong, but all we need whisper is that
they are on the way. Let's.

It is a pity we did it, for she has started up, calling their names;
and there is no one in the room but Nana.

``O Nana, I dreamt my dear ones had come back.''

Nana had filmy eyes, but all she could do was put her paw gently on her
mistress's lap; and they were sitting together thus when the kennel was
brought back. As Mr. Darling puts his head out to kiss his wife, we see
that his face is more worn than of yore, but has a softer expression.

He gave his hat to Liza, who took it scornfully; for she had no
imagination, and was quite incapable of understanding the motives of
such a man. Outside, the crowd who had accompanied the cab home were
still cheering, and he was naturally not unmoved.

``Listen to them,'' he said; ``it is very gratifying.''

``Lots of little boys,'' sneered Liza.

``There were several adults to-day,'' he assured her with a faint flush;
but when she tossed her head he had not a word of reproof for her.
Social success had not spoilt him; it had made him sweeter. For some
time he sat with his head out of the kennel, talking with Mrs. Darling
of this success, and pressing her hand reassuringly when she said she
hoped his head would not be turned by it.

``But if I had been a weak man,'' he said. ``Good heavens, if I had been a
weak man!''

``And, George,'' she said timidly, ``you are as full of remorse as ever,
aren't you?''

``Full of remorse as ever, dearest! See my punishment: living in a
kennel.''

``But it is punishment, isn't it, George? You are sure you are not
enjoying it?''

``My love!''

You may be sure she begged his pardon; and then, feeling drowsy, he
curled round in the kennel.

``Won't you play me to sleep,'' he asked, ``on the nursery piano?'' and as
she was crossing to the day-nursery he added thoughtlessly, ``And shut
that window. I feel a draught.''

``O George, never ask me to do that. The window must always be left open
for them, always, always.''

Now it was his turn to beg her pardon; and she went into the
day-nursery and played, and soon he was asleep; and while he slept,
Wendy and John and Michael flew into the room.

Oh no. We have written it so, because that was the charming arrangement
planned by them before we left the ship; but something must have
happened since then, for it is not they who have flown in, it is Peter
and Tinker Bell.

Peter's first words tell all.

``Quick Tink,'' he whispered, ``close the window; bar it! That's right.
Now you and I must get away by the door; and when Wendy comes she will
think her mother has barred her out; and she will have to go back with
me.''

Now I understand what had hitherto puzzled me, why when Peter had
exterminated the pirates he did not return to the island and leave Tink
to escort the children to the mainland. This trick had been in his head
all the time.

Instead of feeling that he was behaving badly he danced with glee; then
he peeped into the day-nursery to see who was playing. He whispered to
Tink, ``It's Wendy's mother! She is a pretty lady, but not so pretty as
my mother. Her mouth is full of thimbles, but not so full as my
mother's was.''

Of course he knew nothing whatever about his mother; but he sometimes
bragged about her.

He did not know the tune, which was ``Home, Sweet Home,'' but he knew it
was saying, ``Come back, Wendy, Wendy, Wendy''; and he cried exultantly,
``You will never see Wendy again, lady, for the window is barred!''

He peeped in again to see why the music had stopped, and now he saw
that Mrs. Darling had laid her head on the box, and that two tears were
sitting on her eyes.

``She wants me to unbar the window,'' thought Peter, ``but I won't, not
I!''

He peeped again, and the tears were still there, or another two had
taken their place.

``She's awfully fond of Wendy,'' he said to himself. He was angry with
her now for not seeing why she could not have Wendy.

The reason was so simple: ``I'm fond of her too. We can't both have her,
lady.''

But the lady would not make the best of it, and he was unhappy. He
ceased to look at her, but even then she would not let go of him. He
skipped about and made funny faces, but when he stopped it was just as
if she were inside him, knocking.

``Oh, all right,'' he said at last, and gulped. Then he unbarred the
window. ``Come on, Tink,'' he cried, with a frightful sneer at the laws
of nature; ``we don't want any silly mothers;'' and he flew away.

Thus Wendy and John and Michael found the window open for them after
all, which of course was more than they deserved. They alighted on the
floor, quite unashamed of themselves, and the youngest one had already
forgotten his home.

``John,'' he said, looking around him doubtfully, ``I think I have been
here before.''

``Of course you have, you silly. There is your old bed.''

``So it is,'' Michael said, but not with much conviction.

``I say,'' cried John, ``the kennel!'' and he dashed across to look into
it.

``Perhaps Nana is inside it,'' Wendy said.

But John whistled. ``Hullo,'' he said, ``there's a man inside it.''

``It's father!'' exclaimed Wendy.

``Let me see father,'' Michael begged eagerly, and he took a good look.
``He is not so big as the pirate I killed,'' he said with such frank
disappointment that I am glad Mr. Darling was asleep; it would have
been sad if those had been the first words he heard his little Michael
say.

Wendy and John had been taken aback somewhat at finding their father in
the kennel.

``Surely,'' said John, like one who had lost faith in his memory, ``he
used not to sleep in the kennel?''

``John,'' Wendy said falteringly, ``perhaps we don't remember the old life
as well as we thought we did.''

A chill fell upon them; and serve them right.

``It is very careless of mother,'' said that young scoundrel John, ``not
to be here when we come back.''

It was then that Mrs. Darling began playing again.

``It's mother!'' cried Wendy, peeping.

``So it is!'' said John.

``Then are you not really our mother, Wendy?'' asked Michael, who was
surely sleepy.

``Oh dear!'' exclaimed Wendy, with her first real twinge of remorse, ``it
was quite time we came back.''

``Let us creep in,'' John suggested, ``and put our hands over her eyes.''

But Wendy, who saw that they must break the joyous news more gently,
had a better plan.

``Let us all slip into our beds, and be there when she comes in, just as
if we had never been away.''

And so when Mrs. Darling went back to the night-nursery to see if her
husband was asleep, all the beds were occupied. The children waited for
her cry of joy, but it did not come. She saw them, but she did not
believe they were there. You see, she saw them in their beds so often
in her dreams that she thought this was just the dream hanging around
her still.

She sat down in the chair by the fire, where in the old days she had
nursed them.

They could not understand this, and a cold fear fell upon all the three
of them.

``Mother!'' Wendy cried.

``That's Wendy,'' she said, but still she was sure it was the dream.

``Mother!''

``That's John,'' she said.

``Mother!'' cried Michael. He knew her now.

``That's Michael,'' she said, and she stretched out her arms for the
three little selfish children they would never envelop again. Yes, they
did, they went round Wendy and John and Michael, who had slipped out of
bed and run to her.

``George, George!'' she cried when she could speak; and Mr. Darling woke
to share her bliss, and Nana came rushing in. There could not have been
a lovelier sight; but there was none to see it except a little boy who
was staring in at the window. He had had ecstasies innumerable that
other children can never know; but he was looking through the window at
the one joy from which he must be for ever barred.




Chapter XVII.
WHEN WENDY GREW UP


I hope you want to know what became of the other boys. They were
waiting below to give Wendy time to explain about them; and when they
had counted five hundred they went up. They went up by the stair,
because they thought this would make a better impression. They stood in
a row in front of Mrs. Darling, with their hats off, and wishing they
were not wearing their pirate clothes. They said nothing, but their
eyes asked her to have them. They ought to have looked at Mr. Darling
also, but they forgot about him.

Of course Mrs. Darling said at once that she would have them; but Mr.
Darling was curiously depressed, and they saw that he considered six a
rather large number.

``I must say,'' he said to Wendy, ``that you don't do things by halves,'' a
grudging remark which the twins thought was pointed at them.

The first twin was the proud one, and he asked, flushing, ``Do you think
we should be too much of a handful, sir? Because, if so, we can go
away.''

``Father!'' Wendy cried, shocked; but still the cloud was on him. He knew
he was behaving unworthily, but he could not help it.

``We could lie doubled up,'' said Nibs.

``I always cut their hair myself,'' said Wendy.

``George!'' Mrs. Darling exclaimed, pained to see her dear one showing
himself in such an unfavourable light.

Then he burst into tears, and the truth came out. He was as glad to
have them as she was, he said, but he thought they should have asked
his consent as well as hers, instead of treating him as a cypher in his
own house.

``I don't think he is a cypher,'' Tootles cried instantly. ``Do you think
he is a cypher, Curly?''

``No, I don't. Do you think he is a cypher, Slightly?''

``Rather not. Twin, what do you think?''

It turned out that not one of them thought him a cypher; and he was
absurdly gratified, and said he would find space for them all in the
drawing-room if they fitted in.

``We'll fit in, sir,'' they assured him.

``Then follow the leader,'' he cried gaily. ``Mind you, I am not sure that
we have a drawing-room, but we pretend we have, and it's all the same.
Hoop la!''

He went off dancing through the house, and they all cried ``Hoop la!''
and danced after him, searching for the drawing-room; and I forget
whether they found it, but at any rate they found corners, and they all
fitted in.

As for Peter, he saw Wendy once again before he flew away. He did not
exactly come to the window, but he brushed against it in passing so
that she could open it if she liked and call to him. That is what she
did.

``Hullo, Wendy, good-bye,'' he said.

``Oh dear, are you going away?''

``Yes.''

``You don't feel, Peter,'' she said falteringly, ``that you would like to
say anything to my parents about a very sweet subject?''

``No.''

``About me, Peter?''

``No.''

Mrs. Darling came to the window, for at present she was keeping a sharp
eye on Wendy. She told Peter that she had adopted all the other boys,
and would like to adopt him also.

``Would you send me to school?'' he inquired craftily.

``Yes.''

``And then to an office?''

``I suppose so.''

``Soon I would be a man?''

``Very soon.''

``I don't want to go to school and learn solemn things,'' he told her
passionately. ``I don't want to be a man. O Wendy's mother, if I was to
wake up and feel there was a beard!''

``Peter,'' said Wendy the comforter, ``I should love you in a beard;'' and
Mrs. Darling stretched out her arms to him, but he repulsed her.

``Keep back, lady, no one is going to catch me and make me a man.''

``But where are you going to live?''

``With Tink in the house we built for Wendy. The fairies are to put it
high up among the tree tops where they sleep at nights.''

``How lovely,'' cried Wendy so longingly that Mrs. Darling tightened her
grip.

``I thought all the fairies were dead,'' Mrs. Darling said.

``There are always a lot of young ones,'' explained Wendy, who was now
quite an authority, ``because you see when a new baby laughs for the
first time a new fairy is born, and as there are always new babies
there are always new fairies. They live in nests on the tops of trees;
and the mauve ones are boys and the white ones are girls, and the blue
ones are just little sillies who are not sure what they are.''

``I shall have such fun,'' said Peter, with eye on Wendy.

``It will be rather lonely in the evening,'' she said, ``sitting by the
fire.''

``I shall have Tink.''

``Tink can't go a twentieth part of the way round,'' she reminded him a
little tartly.

``Sneaky tell-tale!'' Tink called out from somewhere round the corner.

``It doesn't matter,'' Peter said.

``O Peter, you know it matters.''

``Well, then, come with me to the little house.''

``May I, mummy?''

``Certainly not. I have got you home again, and I mean to keep you.''

``But he does so need a mother.''

``So do you, my love.''

``Oh, all right,'' Peter said, as if he had asked her from politeness
merely; but Mrs. Darling saw his mouth twitch, and she made this
handsome offer: to let Wendy go to him for a week every year to do his
spring cleaning. Wendy would have preferred a more permanent
arrangement; and it seemed to her that spring would be long in coming;
but this promise sent Peter away quite gay again. He had no sense of
time, and was so full of adventures that all I have told you about him
is only a halfpenny-worth of them. I suppose it was because Wendy knew
this that her last words to him were these rather plaintive ones:

``You won't forget me, Peter, will you, before spring cleaning time
comes?''

Of course Peter promised; and then he flew away. He took Mrs. Darling's
kiss with him. The kiss that had been for no one else, Peter took quite
easily. Funny. But she seemed satisfied.

Of course all the boys went to school; and most of them got into Class
III, but Slightly was put first into Class IV and then into Class V.
Class I is the top class. Before they had attended school a week they
saw what goats they had been not to remain on the island; but it was
too late now, and soon they settled down to being as ordinary as you or
me or Jenkins minor. It is sad to have to say that the power to fly
gradually left them. At first Nana tied their feet to the bed-posts so
that they should not fly away in the night; and one of their diversions
by day was to pretend to fall off buses; but by and by they ceased to
tug at their bonds in bed, and found that they hurt themselves when
they let go of the bus. In time they could not even fly after their
hats. Want of practice, they called it; but what it really meant was
that they no longer believed.

Michael believed longer than the other boys, though they jeered at him;
so he was with Wendy when Peter came for her at the end of the first
year. She flew away with Peter in the frock she had woven from leaves
and berries in the Neverland, and her one fear was that he might notice
how short it had become; but he never noticed, he had so much to say
about himself.

She had looked forward to thrilling talks with him about old times, but
new adventures had crowded the old ones from his mind.

``Who is Captain Hook?'' he asked with interest when she spoke of the
arch enemy.

``Don't you remember,'' she asked, amazed, ``how you killed him and saved
all our lives?''

``I forget them after I kill them,'' he replied carelessly.

When she expressed a doubtful hope that Tinker Bell would be glad to
see her he said, ``Who is Tinker Bell?''

``O Peter,'' she said, shocked; but even when she explained he could not
remember.

``There are such a lot of them,'' he said. ``I expect she is no more.''

I expect he was right, for fairies don't live long, but they are so
little that a short time seems a good while to them.

Wendy was pained too to find that the past year was but as yesterday to
Peter; it had seemed such a long year of waiting to her. But he was
exactly as fascinating as ever, and they had a lovely spring cleaning
in the little house on the tree tops.

Next year he did not come for her. She waited in a new frock because
the old one simply would not meet; but he never came.

``Perhaps he is ill,'' Michael said.

``You know he is never ill.''

Michael came close to her and whispered, with a shiver, ``Perhaps there
is no such person, Wendy!'' and then Wendy would have cried if Michael
had not been crying.

Peter came next spring cleaning; and the strange thing was that he
never knew he had missed a year.

That was the last time the girl Wendy ever saw him. For a little longer
she tried for his sake not to have growing pains; and she felt she was
untrue to him when she got a prize for general knowledge. But the years
came and went without bringing the careless boy; and when they met
again Wendy was a married woman, and Peter was no more to her than a
little dust in the box in which she had kept her toys. Wendy was grown
up. You need not be sorry for her. She was one of the kind that likes
to grow up. In the end she grew up of her own free will a day quicker
than other girls.

All the boys were grown up and done for by this time; so it is scarcely
worth while saying anything more about them. You may see the twins and
Nibs and Curly any day going to an office, each carrying a little bag
and an umbrella. Michael is an engine-driver. Slightly married a lady
of title, and so he became a lord. You see that judge in a wig coming
out at the iron door? That used to be Tootles. The bearded man who
doesn't know any story to tell his children was once John.

Wendy was married in white with a pink sash. It is strange to think
that Peter did not alight in the church and forbid the banns.

Years rolled on again, and Wendy had a daughter. This ought not to be
written in ink but in a golden splash.

She was called Jane, and always had an odd inquiring look, as if from
the moment she arrived on the mainland she wanted to ask questions.
When she was old enough to ask them they were mostly about Peter Pan.
She loved to hear of Peter, and Wendy told her all she could remember
in the very nursery from which the famous flight had taken place. It
was Jane's nursery now, for her father had bought it at the three per
cents from Wendy's father, who was no longer fond of stairs. Mrs.
Darling was now dead and forgotten.

There were only two beds in the nursery now, Jane's and her nurse's;
and there was no kennel, for Nana also had passed away. She died of old
age, and at the end she had been rather difficult to get on with; being
very firmly convinced that no one knew how to look after children
except herself.

Once a week Jane's nurse had her evening off; and then it was Wendy's
part to put Jane to bed. That was the time for stories. It was Jane's
invention to raise the sheet over her mother's head and her own, thus
making a tent, and in the awful darkness to whisper:

``What do we see now?''

``I don't think I see anything to-night,'' says Wendy, with a feeling
that if Nana were here she would object to further conversation.

``Yes, you do,'' says Jane, ``you see when you were a little girl.''

``That is a long time ago, sweetheart,'' says Wendy. ``Ah me, how time
flies!''

``Does it fly,'' asks the artful child, ``the way you flew when you were a
little girl?''

``The way I flew? Do you know, Jane, I sometimes wonder whether I ever
did really fly.''

``Yes, you did.''

``The dear old days when I could fly!''

``Why can't you fly now, mother?''

``Because I am grown up, dearest. When people grow up they forget the
way.''

``Why do they forget the way?''

``Because they are no longer gay and innocent and heartless. It is only
the gay and innocent and heartless who can fly.''

``What is gay and innocent and heartless? I do wish I were gay and
innocent and heartless.''

Or perhaps Wendy admits she does see something.

``I do believe,'' she says, ``that it is this nursery.''

``I do believe it is,'' says Jane. ``Go on.''

They are now embarked on the great adventure of the night when Peter
flew in looking for his shadow.

``The foolish fellow,'' says Wendy, ``tried to stick it on with soap, and
when he could not he cried, and that woke me, and I sewed it on for
him.''

``You have missed a bit,'' interrupts Jane, who now knows the story
better than her mother. ``When you saw him sitting on the floor crying,
what did you say?''

``I sat up in bed and I said, ‘Boy, why are you crying?'''

``Yes, that was it,'' says Jane, with a big breath.

``And then he flew us all away to the Neverland and the fairies and the
pirates and the redskins and the mermaids' lagoon, and the home under
the ground, and the little house.''

``Yes! which did you like best of all?''

``I think I liked the home under the ground best of all.''

``Yes, so do I. What was the last thing Peter ever said to you?''

``The last thing he ever said to me was, ‘Just always be waiting for me,
and then some night you will hear me crowing.'''

``Yes.''

``But, alas, he forgot all about me,'' Wendy said it with a smile. She
was as grown up as that.

``What did his crow sound like?'' Jane asked one evening.

``It was like this,'' Wendy said, trying to imitate Peter's crow.

``No, it wasn't,'' Jane said gravely, ``it was like this;'' and she did it
ever so much better than her mother.

Wendy was a little startled. ``My darling, how can you know?''

``I often hear it when I am sleeping,'' Jane said.

``Ah yes, many girls hear it when they are sleeping, but I was the only
one who heard it awake.''

``Lucky you,'' said Jane.

And then one night came the tragedy. It was the spring of the year, and
the story had been told for the night, and Jane was now asleep in her
bed. Wendy was sitting on the floor, very close to the fire, so as to
see to darn, for there was no other light in the nursery; and while she
sat darning she heard a crow. Then the window blew open as of old, and
Peter dropped in on the floor.

He was exactly the same as ever, and Wendy saw at once that he still
had all his first teeth.

He was a little boy, and she was grown up. She huddled by the fire not
daring to move, helpless and guilty, a big woman.

``Hullo, Wendy,'' he said, not noticing any difference, for he was
thinking chiefly of himself; and in the dim light her white dress might
have been the nightgown in which he had seen her first.

``Hullo, Peter,'' she replied faintly, squeezing herself as small as
possible. Something inside her was crying ``Woman, Woman, let go of me.''

``Hullo, where is John?'' he asked, suddenly missing the third bed.

``John is not here now,'' she gasped.

``Is Michael asleep?'' he asked, with a careless glance at Jane.

``Yes,'' she answered; and now she felt that she was untrue to Jane as
well as to Peter.

``That is not Michael,'' she said quickly, lest a judgment should fall on
her.

Peter looked. ``Hullo, is it a new one?''

``Yes.''

``Boy or girl?''

``Girl.''

Now surely he would understand; but not a bit of it.

``Peter,'' she said, faltering, ``are you expecting me to fly away with
you?''

``Of course; that is why I have come.'' He added a little sternly, ``Have
you forgotten that this is spring cleaning time?''

She knew it was useless to say that he had let many spring cleaning
times pass.

``I can't come,'' she said apologetically, ``I have forgotten how to fly.''

``I'll soon teach you again.''

``O Peter, don't waste the fairy dust on me.''

She had risen; and now at last a fear assailed him. ``What is it?'' he
cried, shrinking.

``I will turn up the light,'' she said, ``and then you can see for
yourself.''

For almost the only time in his life that I know of, Peter was afraid.
``Don't turn up the light,'' he cried.

She let her hands play in the hair of the tragic boy. She was not a
little girl heart-broken about him; she was a grown woman smiling at it
all, but they were wet-eyed smiles.

Then she turned up the light, and Peter saw. He gave a cry of pain; and
when the tall beautiful creature stooped to lift him in her arms he
drew back sharply.

``What is it?'' he cried again.

She had to tell him.

``I am old, Peter. I am ever so much more than twenty. I grew up long
ago.''

``You promised not to!''

``I couldn't help it. I am a married woman, Peter.''

``No, you're not.''

``Yes, and the little girl in the bed is my baby.''

``No, she's not.''

But he supposed she was; and he took a step towards the sleeping child
with his dagger upraised. Of course he did not strike. He sat down on
the floor instead and sobbed; and Wendy did not know how to comfort
him, though she could have done it so easily once. She was only a woman
now, and she ran out of the room to try to think.

Peter continued to cry, and soon his sobs woke Jane. She sat up in bed,
and was interested at once.

``Boy,'' she said, ``why are you crying?''

Peter rose and bowed to her, and she bowed to him from the bed.

``Hullo,'' he said.

``Hullo,'' said Jane.

``My name is Peter Pan,'' he told her.

``Yes, I know.''

``I came back for my mother,'' he explained, ``to take her to the
Neverland.''

``Yes, I know,'' Jane said, ``I have been waiting for you.''

When Wendy returned diffidently she found Peter sitting on the bed-post
crowing gloriously, while Jane in her nighty was flying round the room
in solemn ecstasy.

``She is my mother,'' Peter explained; and Jane descended and stood by
his side, with the look in her face that he liked to see on ladies when
they gazed at him.

``He does so need a mother,'' Jane said.

``Yes, I know,'' Wendy admitted rather forlornly; ``no one knows it so
well as I.''

``Good-bye,'' said Peter to Wendy; and he rose in the air, and the
shameless Jane rose with him; it was already her easiest way of moving
about.

Wendy rushed to the window.

``No, no,'' she cried.

``It is just for spring cleaning time,'' Jane said, ``he wants me always
to do his spring cleaning.''

``If only I could go with you,'' Wendy sighed.

``You see you can't fly,'' said Jane.

Of course in the end Wendy let them fly away together. Our last glimpse
of her shows her at the window, watching them receding into the sky
until they were as small as stars.

As you look at Wendy, you may see her hair becoming white, and her
figure little again, for all this happened long ago. Jane is now a
common grown-up, with a daughter called Margaret; and every spring
cleaning time, except when he forgets, Peter comes for Margaret and
takes her to the Neverland, where she tells him stories about himself,
to which he listens eagerly. When Margaret grows up she will have a
daughter, who is to be Peter's mother in turn; and thus it will go on,
so long as children are gay and innocent and heartless.

THE END




*** END OF THE PROJECT GUTENBERG EBOOK PETER PAN ***

***** This file should be named 16-0.txt or 16-0.zip *****
This and all associated files of various formats will be found in:
    https://www.gutenberg.org/1/16/

Updated editions will replace the previous one--the old editions will
be renamed.

Creating the works from print editions not protected by U.S. copyright
law means that no one owns a United States copyright in these works,
so the Foundation (and you!) can copy and distribute it in the
United States without permission and without paying copyright
royalties. Special rules, set forth in the General Terms of Use part
of this license, apply to copying and distributing Project
Gutenberg-tm electronic works to protect the PROJECT GUTENBERG-tm
concept and trademark. Project Gutenberg is a registered trademark,
and may not be used if you charge for an eBook, except by following
the terms of the trademark license, including paying royalties for use
of the Project Gutenberg trademark. If you do not charge anything for
copies of this eBook, complying with the trademark license is very
easy. You may use this eBook for nearly any purpose such as creation
of derivative works, reports, performances and research. Project
Gutenberg eBooks may be modified and printed and given away--you may
do practically ANYTHING in the United States with eBooks not protected
by U.S. copyright law. Redistribution is subject to the trademark
license, especially commercial redistribution.

START: FULL LICENSE

THE FULL PROJECT GUTENBERG LICENSE
PLEASE READ THIS BEFORE YOU DISTRIBUTE OR USE THIS WORK

To protect the Project Gutenberg-tm mission of promoting the free
distribution of electronic works, by using or distributing this work
(or any other work associated in any way with the phrase "Project
Gutenberg"), you agree to comply with all the terms of the Full
Project Gutenberg-tm License available with this file or online at
www.gutenberg.org/license.

Section 1. General Terms of Use and Redistributing Project
Gutenberg-tm electronic works

1.A. By reading or using any part of this Project Gutenberg-tm
electronic work, you indicate that you have read, understand, agree to
and accept all the terms of this license and intellectual property
(trademark/copyright) agreement. If you do not agree to abide by all
the terms of this agreement, you must cease using and return or
destroy all copies of Project Gutenberg-tm electronic works in your
possession. If you paid a fee for obtaining a copy of or access to a
Project Gutenberg-tm electronic work and you do not agree to be bound
by the terms of this agreement, you may obtain a refund from the
person or entity to whom you paid the fee as set forth in paragraph
1.E.8.

1.B. "Project Gutenberg" is a registered trademark. It may only be
used on or associated in any way with an electronic work by people who
agree to be bound by the terms of this agreement. There are a few
things that you can do with most Project Gutenberg-tm electronic works
even without complying with the full terms of this agreement. See
paragraph 1.C below. There are a lot of things you can do with Project
Gutenberg-tm electronic works if you follow the terms of this
agreement and help preserve free future access to Project Gutenberg-tm
electronic works. See paragraph 1.E below.

1.C. The Project Gutenberg Literary Archive Foundation ("the
Foundation" or PGLAF), owns a compilation copyright in the collection
of Project Gutenberg-tm electronic works. Nearly all the individual
works in the collection are in the public domain in the United
States. If an individual work is unprotected by copyright law in the
United States and you are located in the United States, we do not
claim a right to prevent you from copying, distributing, performing,
displaying or creating derivative works based on the work as long as
all references to Project Gutenberg are removed. Of course, we hope
that you will support the Project Gutenberg-tm mission of promoting
free access to electronic works by freely sharing Project Gutenberg-tm
works in compliance with the terms of this agreement for keeping the
Project Gutenberg-tm name associated with the work. You can easily
comply with the terms of this agreement by keeping this work in the
same format with its attached full Project Gutenberg-tm License when
you share it without charge with others.

1.D. The copyright laws of the place where you are located also govern
what you can do with this work. Copyright laws in most countries are
in a constant state of change. If you are outside the United States,
check the laws of your country in addition to the terms of this
agreement before downloading, copying, displaying, performing,
distributing or creating derivative works based on this work or any
other Project Gutenberg-tm work. The Foundation makes no
representations concerning the copyright status of any work in any
country other than the United States.

1.E. Unless you have removed all references to Project Gutenberg:

1.E.1. The following sentence, with active links to, or other
immediate access to, the full Project Gutenberg-tm License must appear
prominently whenever any copy of a Project Gutenberg-tm work (any work
on which the phrase "Project Gutenberg" appears, or with which the
phrase "Project Gutenberg" is associated) is accessed, displayed,
performed, viewed, copied or distributed:

  This eBook is for the use of anyone anywhere in the United States and
  most other parts of the world at no cost and with almost no
  restrictions whatsoever. You may copy it, give it away or re-use it
  under the terms of the Project Gutenberg License included with this
  eBook or online at www.gutenberg.org. If you are not located in the
  United States, you will have to check the laws of the country where
  you are located before using this eBook.

1.E.2. If an individual Project Gutenberg-tm electronic work is
derived from texts not protected by U.S. copyright law (does not
contain a notice indicating that it is posted with permission of the
copyright holder), the work can be copied and distributed to anyone in
the United States without paying any fees or charges. If you are
redistributing or providing access to a work with the phrase "Project
Gutenberg" associated with or appearing on the work, you must comply
either with the requirements of paragraphs 1.E.1 through 1.E.7 or
obtain permission for the use of the work and the Project Gutenberg-tm
trademark as set forth in paragraphs 1.E.8 or 1.E.9.

1.E.3. If an individual Project Gutenberg-tm electronic work is posted
with the permission of the copyright holder, your use and distribution
must comply with both paragraphs 1.E.1 through 1.E.7 and any
additional terms imposed by the copyright holder. Additional terms
will be linked to the Project Gutenberg-tm License for all works
posted with the permission of the copyright holder found at the
beginning of this work.

1.E.4. Do not unlink or detach or remove the full Project Gutenberg-tm
License terms from this work, or any files containing a part of this
work or any other work associated with Project Gutenberg-tm.

1.E.5. Do not copy, display, perform, distribute or redistribute this
electronic work, or any part of this electronic work, without
prominently displaying the sentence set forth in paragraph 1.E.1 with
active links or immediate access to the full terms of the Project
Gutenberg-tm License.

1.E.6. You may convert to and distribute this work in any binary,
compressed, marked up, nonproprietary or proprietary form, including
any word processing or hypertext form. However, if you provide access
to or distribute copies of a Project Gutenberg-tm work in a format
other than "Plain Vanilla ASCII" or other format used in the official
version posted on the official Project Gutenberg-tm website
(www.gutenberg.org), you must, at no additional cost, fee or expense
to the user, provide a copy, a means of exporting a copy, or a means
of obtaining a copy upon request, of the work in its original "Plain
Vanilla ASCII" or other form. Any alternate format must include the
full Project Gutenberg-tm License as specified in paragraph 1.E.1.

1.E.7. Do not charge a fee for access to, viewing, displaying,
performing, copying or distributing any Project Gutenberg-tm works
unless you comply with paragraph 1.E.8 or 1.E.9.

1.E.8. You may charge a reasonable fee for copies of or providing
access to or distributing Project Gutenberg-tm electronic works
provided that:

* You pay a royalty fee of 20% of the gross profits you derive from
  the use of Project Gutenberg-tm works calculated using the method
  you already use to calculate your applicable taxes. The fee is owed
  to the owner of the Project Gutenberg-tm trademark, but he has
  agreed to donate royalties under this paragraph to the Project
  Gutenberg Literary Archive Foundation. Royalty payments must be paid
  within 60 days following each date on which you prepare (or are
  legally required to prepare) your periodic tax returns. Royalty
  payments should be clearly marked as such and sent to the Project
  Gutenberg Literary Archive Foundation at the address specified in
  Section 4, "Information about donations to the Project Gutenberg
  Literary Archive Foundation."

* You provide a full refund of any money paid by a user who notifies
  you in writing (or by e-mail) within 30 days of receipt that s/he
  does not agree to the terms of the full Project Gutenberg-tm
  License. You must require such a user to return or destroy all
  copies of the works possessed in a physical medium and discontinue
  all use of and all access to other copies of Project Gutenberg-tm
  works.

* You provide, in accordance with paragraph 1.F.3, a full refund of
  any money paid for a work or a replacement copy, if a defect in the
  electronic work is discovered and reported to you within 90 days of
  receipt of the work.

* You comply with all other terms of this agreement for free
  distribution of Project Gutenberg-tm works.

1.E.9. If you wish to charge a fee or distribute a Project
Gutenberg-tm electronic work or group of works on different terms than
are set forth in this agreement, you must obtain permission in writing
from the Project Gutenberg Literary Archive Foundation, the manager of
the Project Gutenberg-tm trademark. Contact the Foundation as set
forth in Section 3 below.

1.F.

1.F.1. Project Gutenberg volunteers and employees expend considerable
effort to identify, do copyright research on, transcribe and proofread
works not protected by U.S. copyright law in creating the Project
Gutenberg-tm collection. Despite these efforts, Project Gutenberg-tm
electronic works, and the medium on which they may be stored, may
contain "Defects," such as, but not limited to, incomplete, inaccurate
or corrupt data, transcription errors, a copyright or other
intellectual property infringement, a defective or damaged disk or
other medium, a computer virus, or computer codes that damage or
cannot be read by your equipment.

1.F.2. LIMITED WARRANTY, DISCLAIMER OF DAMAGES - Except for the "Right
of Replacement or Refund" described in paragraph 1.F.3, the Project
Gutenberg Literary Archive Foundation, the owner of the Project
Gutenberg-tm trademark, and any other party distributing a Project
Gutenberg-tm electronic work under this agreement, disclaim all
liability to you for damages, costs and expenses, including legal
fees. YOU AGREE THAT YOU HAVE NO REMEDIES FOR NEGLIGENCE, STRICT
LIABILITY, BREACH OF WARRANTY OR BREACH OF CONTRACT EXCEPT THOSE
PROVIDED IN PARAGRAPH 1.F.3. YOU AGREE THAT THE FOUNDATION, THE
TRADEMARK OWNER, AND ANY DISTRIBUTOR UNDER THIS AGREEMENT WILL NOT BE
LIABLE TO YOU FOR ACTUAL, DIRECT, INDIRECT, CONSEQUENTIAL, PUNITIVE OR
INCIDENTAL DAMAGES EVEN IF YOU GIVE NOTICE OF THE POSSIBILITY OF SUCH
DAMAGE.

1.F.3. LIMITED RIGHT OF REPLACEMENT OR REFUND - If you discover a
defect in this electronic work within 90 days of receiving it, you can
receive a refund of the money (if any) you paid for it by sending a
written explanation to the person you received the work from. If you
received the work on a physical medium, you must return the medium
with your written explanation. The person or entity that provided you
with the defective work may elect to provide a replacement copy in
lieu of a refund. If you received the work electronically, the person
or entity providing it to you may choose to give you a second
opportunity to receive the work electronically in lieu of a refund. If
the second copy is also defective, you may demand a refund in writing
without further opportunities to fix the problem.

1.F.4. Except for the limited right of replacement or refund set forth
in paragraph 1.F.3, this work is provided to you 'AS-IS', WITH NO
OTHER WARRANTIES OF ANY KIND, EXPRESS OR IMPLIED, INCLUDING BUT NOT
LIMITED TO WARRANTIES OF MERCHANTABILITY OR FITNESS FOR ANY PURPOSE.

1.F.5. Some states do not allow disclaimers of certain implied
warranties or the exclusion or limitation of certain types of
damages. If any disclaimer or limitation set forth in this agreement
violates the law of the state applicable to this agreement, the
agreement shall be interpreted to make the maximum disclaimer or
limitation permitted by the applicable state law. The invalidity or
unenforceability of any provision of this agreement shall not void the
remaining provisions.

1.F.6. INDEMNITY - You agree to indemnify and hold the Foundation, the
trademark owner, any agent or employee of the Foundation, anyone
providing copies of Project Gutenberg-tm electronic works in
accordance with this agreement, and any volunteers associated with the
production, promotion and distribution of Project Gutenberg-tm
electronic works, harmless from all liability, costs and expenses,
including legal fees, that arise directly or indirectly from any of
the following which you do or cause to occur: (a) distribution of this
or any Project Gutenberg-tm work, (b) alteration, modification, or
additions or deletions to any Project Gutenberg-tm work, and (c) any
Defect you cause.

Section 2. Information about the Mission of Project Gutenberg-tm

Project Gutenberg-tm is synonymous with the free distribution of
electronic works in formats readable by the widest variety of
computers including obsolete, old, middle-aged and new computers. It
exists because of the efforts of hundreds of volunteers and donations
from people in all walks of life.

Volunteers and financial support to provide volunteers with the
assistance they need are critical to reaching Project Gutenberg-tm's
goals and ensuring that the Project Gutenberg-tm collection will
remain freely available for generations to come. In 2001, the Project
Gutenberg Literary Archive Foundation was created to provide a secure
and permanent future for Project Gutenberg-tm and future
generations. To learn more about the Project Gutenberg Literary
Archive Foundation and how your efforts and donations can help, see
Sections 3 and 4 and the Foundation information page at
www.gutenberg.org

Section 3. Information about the Project Gutenberg Literary
Archive Foundation

The Project Gutenberg Literary Archive Foundation is a non-profit
501(c)(3) educational corporation organized under the laws of the
state of Mississippi and granted tax exempt status by the Internal
Revenue Service. The Foundation's EIN or federal tax identification
number is 64-6221541. Contributions to the Project Gutenberg Literary
Archive Foundation are tax deductible to the full extent permitted by
U.S. federal laws and your state's laws.

The Foundation's business office is located at 809 North 1500 West,
Salt Lake City, UT 84116, (801) 596-1887. Email contact links and up
to date contact information can be found at the Foundation's website
and official page at www.gutenberg.org/contact

Section 4. Information about Donations to the Project Gutenberg
Literary Archive Foundation

Project Gutenberg-tm depends upon and cannot survive without
widespread public support and donations to carry out its mission of
increasing the number of public domain and licensed works that can be
freely distributed in machine-readable form accessible by the widest
array of equipment including outdated equipment. Many small donations
($1 to $5,000) are particularly important to maintaining tax exempt
status with the IRS.

The Foundation is committed to complying with the laws regulating
charities and charitable donations in all 50 states of the United
States. Compliance requirements are not uniform and it takes a
considerable effort, much paperwork and many fees to meet and keep up
with these requirements. We do not solicit donations in locations
where we have not received written confirmation of compliance. To SEND
DONATIONS or determine the status of compliance for any particular
state visit www.gutenberg.org/donate

While we cannot and do not solicit contributions from states where we
have not met the solicitation requirements, we know of no prohibition
against accepting unsolicited donations from donors in such states who
approach us with offers to donate.

International donations are gratefully accepted, but we cannot make
any statements concerning tax treatment of donations received from
outside the United States. U.S. laws alone swamp our small staff.

Please check the Project Gutenberg web pages for current donation
methods and addresses. Donations are accepted in a number of other
ways including checks, online payments and credit card donations. To
donate, please visit: www.gutenberg.org/donate

Section 5. General Information About Project Gutenberg-tm electronic works

Professor Michael S. Hart was the originator of the Project
Gutenberg-tm concept of a library of electronic works that could be
freely shared with anyone. For forty years, he produced and
distributed Project Gutenberg-tm eBooks with only a loose network of
volunteer support.

Project Gutenberg-tm eBooks are often created from several printed
editions, all of which are confirmed as not protected by copyright in
the U.S. unless a copyright notice is included. Thus, we do not
necessarily keep eBooks in compliance with any particular paper
edition.

Most people start at our website which has the main PG search
facility: www.gutenberg.org

This website includes information about Project Gutenberg-tm,
including how to make donations to the Project Gutenberg Literary
Archive Foundation, how to help produce our new eBooks, and how to
subscribe to our email newsletter to hear about new eBooks.


