\chapter{The Children are Carried Off}

\lettrine{T}{he} pirate attack had been a complete surprise: a sure proof that the
unscrupulous Hook had conducted it improperly, for to surprise redskins
fairly is beyond the wit of the white man.

By all the unwritten laws of savage warfare it is always the redskin
who attacks, and with the wiliness of his race he does it just before
the dawn, at which time he knows the courage of the whites to be at its
lowest ebb. The white men have in the meantime made a rude stockade on
the summit of yonder undulating ground, at the foot of which a stream
runs, for it is destruction to be too far from water. There they await
the onslaught, the inexperienced ones clutching their revolvers and
treading on twigs, but the old hands sleeping tranquilly until just
before the dawn. Through the long black night the savage scouts
wriggle, snake-like, among the grass without stirring a blade. The
brushwood closes behind them, as silently as sand into which a mole has
dived. Not a sound is to be heard, save when they give vent to a
wonderful imitation of the lonely call of the coyote. The cry is
answered by other braves; and some of them do it even better than the
coyotes, who are not very good at it. So the chill hours wear on, and
the long suspense is horribly trying to the paleface who has to live
through it for the first time; but to the trained hand those ghastly
calls and still ghastlier silences are but an intimation of how the
night is marching.

That this was the usual procedure was so well known to Hook that in
disregarding it he cannot be excused on the plea of ignorance.

The Piccaninnies, on their part, trusted implicitly to his honour, and
their whole action of the night stands out in marked contrast to his.
They left nothing undone that was consistent with the reputation of
their tribe. With that alertness of the senses which is at once the
marvel and despair of civilised peoples, they knew that the pirates
were on the island from the moment one of them trod on a dry stick; and
in an incredibly short space of time the coyote cries began. Every foot
of ground between the spot where Hook had landed his forces and the
home under the trees was stealthily examined by braves wearing their
mocassins with the heels in front. They found only one hillock with a
stream at its base, so that Hook had no choice; here he must establish
himself and wait for just before the dawn. Everything being thus mapped
out with almost diabolical cunning, the main body of the redskins
folded their blankets around them, and in the phlegmatic manner that is
to them, the pearl of manhood squatted above the children's home,
awaiting the cold moment when they should deal pale death.

Here dreaming, though wide-awake, of the exquisite tortures to which
they were to put him at break of day, those confiding savages were
found by the treacherous Hook. From the accounts afterwards supplied by
such of the scouts as escaped the carnage, he does not seem even to
have paused at the rising ground, though it is certain that in that
grey light he must have seen it: no thought of waiting to be attacked
appears from first to last to have visited his subtle mind; he would
not even hold off till the night was nearly spent; on he pounded with
no policy but to fall to. What could the bewildered scouts do, masters
as they were of every war-like artifice save this one, but trot
helplessly after him, exposing themselves fatally to view, while they
gave pathetic utterance to the coyote cry.

Around the brave Tiger Lily were a dozen of her stoutest warriors, and
they suddenly saw the perfidious pirates bearing down upon them. Fell
from their eyes then the film through which they had looked at victory.
No more would they torture at the stake. For them the happy
hunting-grounds was now. They knew it; but as their father's sons they
acquitted themselves. Even then they had time to gather in a phalanx
that would have been hard to break had they risen quickly, but this
they were forbidden to do by the traditions of their race. It is
written that the noble savage must never express surprise in the
presence of the white. Thus terrible as the sudden appearance of the
pirates must have been to them, they remained stationary for a moment,
not a muscle moving; as if the foe had come by invitation. Then,
indeed, the tradition gallantly upheld, they seized their weapons, and
the air was torn with the war-cry; but it was now too late.

It is no part of ours to describe what was a massacre rather than a
fight. Thus perished many of the flower of the Piccaninny tribe. Not
all unavenged did they die, for with Lean Wolf fell Alf Mason, to
disturb the Spanish Main no more, and among others who bit the dust
were Geo. Scourie, Chas. Turley, and the Alsatian Foggerty. Turley fell
to the tomahawk of the terrible Panther, who ultimately cut a way
through the pirates with Tiger Lily and a small remnant of the tribe.

To what extent Hook is to blame for his tactics on this occasion is for
the historian to decide. Had he waited on the rising ground till the
proper hour he and his men would probably have been butchered; and in
judging him it is only fair to take this into account. What he should
perhaps have done was to acquaint his opponents that he proposed to
follow a new method. On the other hand, this, as destroying the element
of surprise, would have made his strategy of no avail, so that the
whole question is beset with difficulties. One cannot at least withhold
a reluctant admiration for the wit that had conceived so bold a scheme,
and the fell genius with which it was carried out.

What were his own feelings about himself at that triumphant moment?
Fain would his dogs have known, as breathing heavily and wiping their
cutlasses, they gathered at a discreet distance from his hook, and
squinted through their ferret eyes at this extraordinary man. Elation
must have been in his heart, but his face did not reflect it: ever a
dark and solitary enigma, he stood aloof from his followers in spirit
as in substance.

The night's work was not yet over, for it was not the redskins he had
come out to destroy; they were but the bees to be smoked, so that he
should get at the honey. It was Pan he wanted, Pan and Wendy and their
band, but chiefly Pan.

Peter was such a small boy that one tends to wonder at the man's hatred
of him. True he had flung Hook's arm to the crocodile, but even this
and the increased insecurity of life to which it led, owing to the
crocodile's pertinacity, hardly account for a vindictiveness so
relentless and malignant. The truth is that there was a something about
Peter which goaded the pirate captain to frenzy. It was not his
courage, it was not his engaging appearance, it was not---. There is no
beating about the bush, for we know quite well what it was, and have
got to tell. It was Peter's cockiness.

This had got on Hook's nerves; it made his iron claw twitch, and at
night it disturbed him like an insect. While Peter lived, the tortured
man felt that he was a lion in a cage into which a sparrow had come.

The question now was how to get down the trees, or how to get his dogs
down? He ran his greedy eyes over them, searching for the thinnest
ones. They wriggled uncomfortably, for they knew he would not scruple
to ram them down with poles.

In the meantime, what of the boys? We have seen them at the first clang
of the weapons, turned as it were into stone figures, open-mouthed, all
appealing with outstretched arms to Peter; and we return to them as
their mouths close, and their arms fall to their sides. The pandemonium
above has ceased almost as suddenly as it arose, passed like a fierce
gust of wind; but they know that in the passing it has determined their
fate.

Which side had won?

The pirates, listening avidly at the mouths of the trees, heard the
question put by every boy, and alas, they also heard Peter's answer.

``If the redskins have won,'' he said, ``they will beat the tom-tom; it is
always their sign of victory.''

Now Smee had found the tom-tom, and was at that moment sitting on it.
``You will never hear the tom-tom again,'' he muttered, but inaudibly of
course, for strict silence had been enjoined. To his amazement Hook
signed him to beat the tom-tom, and slowly there came to Smee an
understanding of the dreadful wickedness of the order. Never, probably,
had this simple man admired Hook so much.

Twice Smee beat upon the instrument, and then stopped to listen
gleefully.

``The tom-tom,'' the miscreants heard Peter cry; ``an Indian victory!''

The doomed children answered with a cheer that was music to the black
hearts above, and almost immediately they repeated their good-byes to
Peter. This puzzled the pirates, but all their other feelings were
swallowed by a base delight that the enemy were about to come up the
trees. They smirked at each other and rubbed their hands. Rapidly and
silently Hook gave his orders: one man to each tree, and the others to
arrange themselves in a line two yards apart.
